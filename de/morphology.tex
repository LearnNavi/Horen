\nchapter{Morphologie}

\section{Das Substantiv}

\subsection{F\"alle} 
Die Fallendungen des Na’vi \"andern sich, je nachdem ob das Wort mit einem
Konsonanten, einem Vokal oder einem Diphthong endet.\footnote{Die von Frommer
verwendeten Fallbezeichnungen spiegeln die von Bernard Comrie verwendete
Terminologie, die dieser in seinen Schriften zu Ergativsprachen verwendet, wider.
In den meisten linguistischen Schriften wird Frommers Subjektiv (oder Nominativ)
als der Intransitiv bezeichnet, der Agens entspricht dem Ergativ, und der Patiens
ist der Akkusativ.}\index{Substantiv!Deklination}

\begin{center}
\begin{tabular}{lccc}
 & Vokal  & Konsonant & Diphthong \\
\hline
Nominativ & \multicolumn{3}{c}{\textit{nichts}} \\
Agens & \N{-l} & \N{-\`il} & \N{-\`il} \\
Patiens & \N{-t}, \N{-ti} & \N{-it}, \N{-ti} & \N{-it}, \N{-ti} \\
Dativ & \N{-r}, \N{-ru} & \N{-ur} & \N{-ru}, \N{-ur}\\
Genitiv & \N{-y\"a}, \N{-\"a} & \N{-\"a} & \N{-\"a} \\
Thematisch  & \N{-ri} & \N{-\`iri} & \N{-ri}  \\
\end{tabular}\end{center}

\noindent\LNWiki{24/3/2010}{http://wiki.learnnavi.org/index.php/Canon/2010/March-June\%23Declension_with_Diphthongs_and_Deixis}

% For gen. with i: http://forum.learnnavi.org/language-updates/genitive-case-refinement-declension-of-tsaw/msg150927/#msg150927

\subsubsection{} Im Regelfalle ist die Genitivendung \N{-y\"a}, aber nach
den Vokalen \N{o} und \N{u} ist sie lediglich \N{-ä}. Somit wird \N{tsulf\"atu} zu
\N{tsulf\"atu\"a}, \N{Na’vi} zu \N{Na’viy\"a} und \N{l\`i’fya} zu \N{l\`i’fyay\"a}.

\subsubsection{} Substantive mit der Endung \N{-ia} \"andern ihre Endung
im Genitiv zu \N{-i\"a}, wie in \N{soai\"a}, abgeleitet von \N{soaia}.
% http://naviteri.org/2011/05/some-miscellaneous-vocabulary/
% This used to be covered by a single example in following "case
% refinement" section link; later generalized.

\subsubsection{} Zus\"atzlich zu einigen Pronomen (\horenref{morph:pron:irreg-gen})
gibt es einige wenige Substantive mit unregelm\"a\ss{}igen Genitiven, z. B.
\N{Omatikayaä}, abgeleitet von \N{Omatikaya}.
% http://forum.learnnavi.org/language-updates/genitive-case-refinement-declension-of-tsaw/msg150927/#msg150927

\subsubsection{} W\"orter, die auf einem Pseudovokal (\N{ll} und \N{rr}) enden,
erhalten die Endung f\"ur Konsonanten: \N{trr-\"a}, \N{’ewll-it}.
\index{Pseudovokal!Deklination}\label{morph:decl:pseudovowel}

\subsubsection{} Manche W\"orter, die auf einen Diphthong enden, sieht man mit
Deklinationen, die man eigentlich bei einem Wort, das auf einen Vokal endet, erwartet.
\Npawl{txo tsive’a ayngal keyeyt}, \D{Wenn du Fehler findest}.

\subsubsection{} Die Variation zwischen den langen und kurzen Endungen im Patiens und dem
Dativ scheinen im Gro\ss{}en und Ganzen eine Frage des Stils und des Wohlklangs zu sein.

\subsection{Das Indefinitiv-o} Einem Substantiv kann das Indefinitv-Suffix \N{-o}
("`(irgend)einer, (irgend)etwas"') angeh\"angt werden. Etwaige Fallendungen folgen dem
\N{-o}.
\index{-o@\textbf{-o}}\index{Unbestimmtes Substantiv}
\LNWiki{14/3/2010}{http://wiki.learnnavi.org/index.php/Canon/2010/March-June\%23
A_Collection}
\NTeri{5/9/2011}{http://naviteri.org/2011/09/\%E2\%80\%9Cby-the-way-what-are-you-reading\%E2\%80\%9D/comment-page-1/\%23comment-1093}

\subsection{Numerus} Substantive im Na’vi k\"onnen im Singular, Dual, Trial und dem Plural
(d. h. vier oder mehr) stehen. Der Numerus wird durch Pr\"afixe angegeben, die allesamt
Lenition zur Folge haben.\index{Dual}\index{Trial}\index{Plural}

\begin{center}
\begin{tabular}{lrl}
Dual & \N{me+} & \N{mefo} ($<$ \N{me+} $+$ \N{po}) \\
Trial & \N{pxe+} & \N{pxehilvan} ($<$ \N{pxe+} $+$ \N{kilvan}) \\
Plural & \N{ay+} & \N{ayswizaw} \\
\end{tabular}
\end{center}

\subsubsection{} Der Pluralpr\"afix darf \textit{nur dann} fallengelassen werden, wenn
Lenition erfolgt. So ist der Plural von \N{prrnen} entweder \N{ayfrrnen} oder
die Kurzform \N{frrnen} (aber beachten Sie dazu \horenref{syn:adp:short-plural}).
\footnote{Ausnahme: \N{’u}, \D{Sache, Gegenstand} hat keinen kurzen Plural,
sondern erscheint \textit{immer} als \N{ayu}.\index{'u@\textbf{'u}!Kein kurzer Plural}}
Die Dual- und Trialpr\"afixe werden auf diese Art und Weise niemals fallengelassen.
\index{Plural!Kurz} \label{morph:short-plural}
\LanguageLog

\subsubsection{} Wenn ein Wort mit \N{e} oder \N{’e} beginnt, dann verschmilzt
das resultierende \N{*ee} zu einem einzigen \N{e}, wodurch z. B. aus \N{me+} $+$
\N{’eveng} \N{meveng} entsteht. Siehe auch \horenref{l-and-s:phonotactics:nsc}.


\section{Das Pronomen}

\subsection{Belebtheit} \QUAESTIO{Eine kurze Umschreibung der Belebtheitshierarchie
k\"onnte hier hilfreich sein. Gilt eine Wanze als belebt? Kurze Erl\"auterung und
Verweis auf den Satzbau?}\index{Belebtheit}

\subsection{Die grundlegenden Pronomen}
Die Pronomen verwenden dieselben Fallendungen wie Substantive.
\begin{center}
\begin{tabular}{rllll}
Person        & Singular & Dual & Trial & Plural \\ 
\hline
1. exklusiv   & \N{\ACC{o}e}  & \N{m\ACC{o}e}  & \N{px\ACC{o}e}   & \N{ay\ACC{o}e} \\
1. inklusiv   & —      & \N{o\ACC{e}ng} & \N{px\ACC{o}eng} & \N{ayo\ACC{e}ng}, \N{aw\ACC{nga}} \\
2.            & \N{nga} & \N{me\ACC{nga}} & \N{pxe\ACC{nga}} & \N{ay\ACC{nga}} \\
3. belebt     & \N{po}  & \N{me\ACC{fo}} & \N{pxe\ACC{fo}}  & \N{ay\ACC{fo}, fo} \\
3. unbelebt   & \N{\ACC{tsa}’u}, \N{tsaw} & \N{me\ACC{sa}’u} & \N{pxe\ACC{sa}'u} & \N{ay\ACC{sa}’u, sa’u} \\
Reflexiv      & \N{sno} & — & — & — \\
\end{tabular}
\end{center}

\subsubsection{} Im allt\"aglichen Sprachgebrauch \"andert sich die Aussprache f\"ur
die Wurzel f\"ur die erste Person, \N{oe}, zu \N{we}, wenn sie nicht am Wortende
steht, so auch \N{wel} f\"ur \N{oel} oder \N{weru} f\"ur \N{oeru}. Dies geschieht jedoch
nicht f\"ur die Formen des Dual und des Trial, \N{moe} und \N{pxoe}, was eine ung\"ultige
Kombinantion von Konsonanten am Wortanfang zur Folge h\"atte, beispielsweise *\N{mwel}.
Diese Betonung wird durch eine Betonungsmarkierung unter dem \N{e} angezeigt.\label{morph:pron:oe-we}

\subsubsection{}Die Pronomen der ersten Person au\ss{}er dem des Singulars sind entweder
exklusiv (schlie\ss{}t die angesprochene Person aus) oder inklusiv (bezieht die
angesprochene Person mit ein). Die Endung f\"ur die Inklusion, \N{-ng}, stammt von
\N{nga}, was zur G\"anze in Erscheinung tritt, wenn Fallendungen angeh\"angt werden.
Der Agens von \N{oeng} ist \N{oengal} und nicht *\N{oeng\`il}.

\subsubsection{} \N{Ayoeng} hat die Kurzform \N{aw\uline{nga}}. Beide k\"onnen nach
Belieben eingesetzt werden, obwohl \N{awnga} h\"aufiger ist.\index{ayoeng@\textbf{ayoeng}}\index{awnga@\textbf{awnga}}

\subsubsection{} Die belebte Form der dritten Person \N{po} unterscheidet das
Geschlecht nicht und funktioniert sowohl mit "`er"' als auch mit "`sie"' im Deutschen.
Allerdings gibt es auch die geschlechtsspezifischen Formen \N{po\ACC{an}} und
\N{po\ACC{e}}, die regul\"ar dekliniert werden, aber \"uber keine Pluralform
verf\"ugen. Siehe auch \horenref{syn:pron:gender} zur Anwendung.\label{morph:pron:gender}

\subsubsection{} Genitive: \N{pey\"a} $<$ \N{po}; \N{ngey\"a} $<$ \N{nga}
(einschlie\ss{}lich \N{awngey\"a} $<$ \N{awnga}), \N{sney\"a} $<$ \N{sno}.
\index{nga@\textbf{nga}!Genitiv \textbf{ngey\"a}}\label{morph:pron:irreg-gen}
\index{po@\textbf{po}!Genitiv \textbf{pey\"a}}
\index{sno@\textbf{sno}!Genitiv \textbf{sney\"a}}
\index{awnga@\textbf{awnga}!Genitiv \textbf{awngey\"a}}

\subsubsection{} In der informellen und verk\"urzten milit\"arischen Sprechweise
kann das End-\N{\"a} des Genitivs von Pronomen entfallen, beispielsweise
\N{ngey ’upxaret}.\label{morph:pron:gen-clipped} \index{Genitiv!Kurzform von Pronomen}
\index{Pronomen!Verk\"urzte Form}

\subsubsection{} Die unbelebte dritte Person, \N{tsa’u}, ist einfach das
Demonstrativpronomen "`das"'. Wenn Fallendungen oder Adpositionen angeh\"angt werden,
dann kann es wahlweise als \N{tsa’u-} (am formellsten), \N{tsaw-} oder \N{tsa-}
(am informellsten) auftreten. Der Genitiv ist \N{tsey\"a}.\label{morph:pron:tsa}
%http://naviteri.org/2011/08/new-vocabulary-clothing/comment-page-1/#comment-917

\subsubsection{} Das Reflexivpronomen \N{sno} wird nicht nach dem Numerus ver\"andert.
\index{sno@\textbf{sno}!Kein Numerus}

\subsubsection{} Das unbestimmte Pronomen der belebten dritten Person ist \N{fko}.

\subsection{Zeremonielle/ehrerbietende Pronomen}

\begin{center}
\begin{tabular}{rllll}
      & Singular & Dual & Trial & Plural \\ 
\hline
1. exklusiv & \N{\ACC{o}he}  & \N{\ACC{mo}he}  & \N{\ACC{pxo}he}   & \N{ay\ACC{o}he} \\
2.          & \N{nge\ACC{nga}} & \N{menge\ACC{nga}} & \N{pxenge\ACC{nga}} & \N{aynge\ACC{nga}} \\
\end{tabular}
\end{center}\index{Pronomen!ehrerbietende}\label{morph:hon-pron}

\subsubsection{} Um die inklusiven Formen der ersten Person darzustellen, verwenden
Sie getrennte Pronomen (\N{ohe ngengas\`i} (mit angeh\"angtem \N{s\`i}, \D{und}).
\QUAESTIO{Im Film taucht offenbar \N{ohengey\"a} auf.}

%\QUAESTIO{\subsection{-po} We have \N{'awpo} and \N{fìpo}.  What about \N{tsapo}?}

\subsection{Lahe} Wird dieses Wort als Pronomen verwendet, dann hat das Adjektiv
\N{lahe} einen unregelm\"a\ss{}igen Dativ im Plural: \N{aylaru}.
\index{lahe@\textbf{lahe}!Deklination}\label{morph:lahe:dat-pl}



\section{Vorsilben}

\noindent Vorsilben sind einem Adjektiv \"ahnliche Substantivpr\"afixe.\index{Vorsilben}

\subsection{F\`i-} Diese Vorsilbe steht für die proximale Deixis, \D{diese/-r/-s}. Folgt
ihr der Pluralpr\"afix \N{ay+}, dann verschmelzen beide zu \N{fay+}, \D{diese}.
\QUAESTIO{Jedoch haben wir \N{f\`iay+} mindestens einmal bei Frommer gesehen:
\N{oel foru f\`iayl\`i’ut tol\`ing a krr, kxawm oe harmahängaw,} 26. Januar}\label{morph:prenoun:fi}
\index{fi\`i-@\textbf{f\`i-}}\index{fay+@\textbf{fay+}}

\subsubsection{} Einige Substantive und Adjektive werden zu Adverben, indem sie sich
mit \N{f\`i-} verbinden, beispielsweise \N{f\`itrr}, \D{heute} und \N{f\`itxan}, \D{so (viel)}.

\subsection{Tsa-} Dies ist die Ferndeixis, \D{jene/-r/-s}. Folgt ihr der Pluralpr\"afix
\N{ay+}, dann verschmelzen sie zu \N{tsay+}, \D{jene}.
\index{tsa-@\textbf{tsa-}} \index{tsay+@\textbf{tsay+}}

\subsection{-Pe+} \label{morph:pre:pe} Diese Fragevorsilbe bedeutet \D{was, welche/-r/-s},
wie in \N{pel\`i’u}, \D{welches Wort}? -Pe+ ist insofern ungew\"ohnlich, als da\ss{} es sowohl
als Pr\"afix (\N{pel\`i’u}) als auch als Suffix (\N{l\`i’upe}) verwendet werden kann.
Wird es vorangestellt, dann wird das folgende Wort leniert.
Folgt diesem Pr\"afix der Pluralpr\"afix, verschmelzen sie zu \N{pay+}.

\subsection{Fra-} Diese Vorsilbe bedeutet \D{alle/-s, jede/-r/-s}. \QUAESTIO{Folgt der
Pluralpr\"afix \N{ay+}, dann verschmelzen sie zu \N{fray+}.}
\index{fra-@\textbf{fra-}} \index{fray+@\textbf{fray+}}

\subsection{Fne-} Dieser Pr\"afix bedeutet \D{eine Art von}.\index{fne-@\textbf{fne-}}

\subsubsection{} Dieser Pr\"afix ist mit dem Substantiv \N{fnel} verwandt, welches
ebenfalls \D{Art, Sorte} bedeutet. Es kann zusammen mit einem Substantiv im Genitiv
auftreten, wodurch es dieselbe Bedeutung wie der Pr\"afix bekommt. \N{Tsafnel syulang\"a}
und \N{tsafnesyulang} bedeuten beide \D{jene Art von Blume}.\index{fnel@\textbf{fnel}}

\subsection{Verschmelzung} Endet eine Vorsilbe auf dem gleichen Vokal, mit dem das
folgende Wort beginnt, dann verschmelzen die Vokale, wie in \N{tsatan}, \D{jenes Licht},
abgeleitet von \N{tsa-atan} (\horenref{l-and-s:phonotactics:precontract}).
\index{Vorsilbe!Verschmelzung}\label{morph:prenoun:contraction}

\subsection{Kombinationen} Die Vorsilben k\"onnen auf einem Wort kombiniert in folgender
Reihenfolge kombiniert werden: \index{Vorsilbe!Kombinationen}

\begin{center}
\begin{tabular}{cccccc}
\N{f\`i-} \\
\N{tsa-} & \N{fra-} & Numerus & \N{fne-} & das Substantiv & \N{-pe} \\
\N{pe+}
\end{tabular}
\end{center}

\noindent Nur eine Vorsilbe aus jeder Spalte darf eingesetzt werden, und der Frageaffix
wird selbstverst\"andlich nur einmal verwendet. \QUAESTIO{Die vollst\"andigen Details
f\"ur diese Reihenfolge sind f\"ur \N{fra-} noch nicht best\"atigt.}

\subsubsection{} Kurze Plurale (\horenref{morph:short-plural}) werden nicht zusammen mit
den deiktischen Vorsilben verwendet; \N{tsaytele}, \D{jene Angelegenheiten}, jedoch niemals
*\N{tsatele} (Singular: \N{txele}).\index{Plural!Kurzform!nicht zusammen mit Vorsilben}


\section{Entsprechungen}

\noindent Die Demonstrativpronomen und bestimmte g\"angige Adverben der Zeit, der
Art und Weise sowie des Ortes sind letztlich mit Vorsilben kombinierte Substantive.
Jedoch ist dieses System nicht vollst\"andig regul\"ar.

% This is what I get for attaching several words to the same footnote.
\addtocounter{footnote}{1}
\newcounter{coraccent}\setcounter{coraccent}{\value{footnote}}

\begin{center}
\begin{tabular}{rllllll}% name person thing time place manner
 & Person & Sache & Handlung & Zeit & Ort & Art und Weise \\
\hline
\multirow{2}{*}{dies} & \N{\ACC{f\`i}po} & \N{f\`i\ACC{’u}} &
  \N{f\`i\ACC{kem}} & \N{set} & \N{f\`i\ACC{tseng}(e)} & \N{f\`i\ACC{fya}}  \\ 
 & \D{diese/-r} & \D{dieses} & \D{jetzt} & \D{jetzt} &
  \D{hier} & \D{so, auf diese Weise} \\
\multirow{2}{*}{jenes} & \N{\ACC{tsa}tu} & \N{\ACC{tsa}’u} & \N{tsakem}\footnotemark[\value{coraccent}] & \N{tsa\ACC{krr}} &
   \N{tsatseng}\footnotemark[\value{coraccent}] & \N{\ACC{tsa}fya} \\
 & \D{jene/-r} & \D{jenes} & \D{jene} & \D{dann} &
  \D{dort} & \D{auf jene Weise} \\
\multirow{2}{*}{alle} & \N{\ACC{fra}po} & \N{\ACC{fra}’u} & --- &
  \N{\ACC{fra}krr} & \N{\ACC{fra}tseng} & ---  \\
 & \D{jede/-r} & \D{alles} &  & \D{immer} & \D{\"uberall}\\
\multirow{2}{*}{nein} & \N{\ACC{kaw}tu} & \N{\ACC{ke}’u} & \N{\ACC{ke}kem} &
  \N{\ACC{kaw}krr} & \N{\ACC{kaw}tseng} & --- \\
 & \D{niemand} & \D{nichts} & \D{nichts} & \D{nie} & \D{nirgends} \\
\end{tabular}
\end{center}\label{morph:correlatives}
\footnotetext[\value{coraccent}]{Die betonte Silbe ist wahlfrei.}
\LNWiki{18/5/2011}{http://wiki.learnnavi.org/index.php/Canon/2011/April-December\%23Kawtseng.2C_tsapo_and_prefixes}
\NTeri{24/7/2011}{http://naviteri.org/2011/07/txantsana-ultxa-mi-siatll-great-meeting-in-seattle/comment-page-1/\%23comment-845} % kekem

\subsubsection{} \QUAESTIO{Die Plurale hierf\"ur sind ein wenig unkonventionell.
\N{tsa’u} kommt von \N{tsa-} und \N{’u}, und der Plural ist \N{(ay)sa’u}.
Best\"atigt, aber detailliertere Informationen w\"aren gut. Wie funktioniert
das mit \N{tsapo}?}

\subsubsection{} F\"ur die Formen von \N{tsa’u} siehe auch \horenref{morph:pron:tsa}.

\subsection{Fragen} Wie bei Substantiven kann der Frageaffix \N{-pe+} sowohl ein lenisierender
Pr\"afix als auch ein Suffix sein.

\begin{center}
\begin{tabular}{rl}
Wer? & \N{pe\ACC{su}}, \N{\ACC{tu}pe} \\
Was (Sache)? & \N{pe\ACC{u}}, \N{\ACC{’u}pe} \\
Was (Handlung)? & \N{pe\ACC{hem}} \N{\ACC{kem}pe} \\
Wann? & \N{pe\ACC{hrr}}, \N{\ACC{krr}pe} \\
\end{tabular}
\hskip2em
\begin{tabular}{rll}
Wo? & \N{pe\ACC{seng}}, \N{\ACC{tseng}pe} \\
Wie? & \N{pe\ACC{fya}}, \N{\ACC{fya}pe} \\
Warum? & \N{pe\ACC{lun}}, \N{\ACC{lum}pe} \\
Welche Art (von)? & \N{pe\ACC{fnel}}, \N{\ACC{fne}pe}\\
\end{tabular}
\end{center}

\subsubsection{} Das Fragewort f\"ur Person, \N{tupe} / \N{pesu}, \D{wer},
verf\"ugt \"uber eine gewaltige Anzahl an geschlechtsspezifischen und
Nichtsingularformen:

\begin{center}
\begin{tabular}{lccc}
 & Allgemein & m\"annlich & weiblich \\
\hline
Singular & \N{pe\ACC{su}}, \N{\ACC{tu}pe} & 
           \N{pe\ACC{stan}}, \N{tu\ACC{tam}pe} &
           \N{pe\ACC{ste}}, \N{tu\ACC{te}pe} \\
Dual     & \N{pem\ACC{su}}, \N{me\ACC{su}pe} & 
           \N{pem\ACC{stan}}, \N{me\ACC{stam}pe} &
           \N{pem\ACC{ste}}, \N{me\ACC{ste}pe} \\
Trial    & \N{pep\ACC{su}}, \N{pxe\ACC{su}pe} & 
           \N{pep\ACC{stan}}, \N{pxe\ACC{stam}pe} &
           \N{pep\ACC{ste}}, \N{pxe\ACC{ste}pe} \\
Plural   & \N{pay\ACC{su}}, \N{(ay)\ACC{su}pe} & 
           \N{pay\ACC{stan}}, \N{(ay)\ACC{stam}pe} &
           \N{pay\ACC{ste}}, \N{(ay)\ACC{ste}pe} \\
\end{tabular}
\end{center}

\noindent Die Nichtsingularformen von \N{pehem} / \N{kempe} folgen einem
\"ahnlichen Muster:

\begin{center}
\begin{tabular}{lc}
Singular & \N{pe\ACC{hem}}, \N{\ACC{kem}pe} \\
Dual & \N{pem\ACC{hem}}, \N{me\ACC{hem}pe} \\
Trial & \N{pep\ACC{hem}}, \N{pxe\ACC{hem}pe} \\
Plural & \N{pay\ACC{hem}}, \N{(ay)\ACC{hem}pe} \\
\end{tabular}
\end{center}

\subsection{F\`i’u und Tsaw in Gliedsatzsubstantivierungen} Das Demonstrativpronomen
\N{f\`i’u} und das unbelebte Pronomen \N{tsaw} werden zusammen mit dem Attributivpartikel
\N{a} eingesetzt, um Glieds\"atze zu substantivieren (\horenref{syn:clause-nom}).
Folgt das Attributivpartikel auf bestimmte Fallformen des Pronomens, verschmelzen sie.
\label{morph:fwa-tsawa}
% http://forum.learnnavi.org/language-updates/txelanit-hivawl/

\begin{center}
\begin{tabular}{rcc}
Fall & \N{F\`i'u} Verschmelzung & \N{Tsaw} Verschmelzung \\
\hline
Nominativ & \N{fwa} ($<$ \N{f\`i’u a}) & \N{\ACC{tsa}wa} \\
Agens & \N{\ACC{fu}la} ($<$ \N{f\`i’ul a}) & \N{\QUAESTIO{tsala}} \\
Patiens & \N{\ACC{fu}ta} ($<$ \N{f\`i’ut a}) & \N{\ACC{tsa}ta} \\
Thematisch & \N{\ACC{fu}ria} ($<$ \N{f\`i’uri a}) & \N{\ACC{tsa}ria} \\
\end{tabular}
\end{center}
\index{fwa@\textbf{fwa}}\index{tsawa@\textbf{tsawa}}
\index{fula@\textbf{fula}}
\index{futa@\textbf{futa}}\index{tsata@\textbf{tsata}}
\index{furia@\textbf{furia}}\index{tsaria@\textbf{tsaria}}
\LNWiki{18/6/2010}{http://wiki.learnnavi.org/index.php/Canon/2010/March-June\%23The_contrast_between_fwa.2Ftsawa.2C_furia.2Ftsaria}

\subsection{Fmawn und T\`i’eyng in Gliedsatzsubstantivierungen} W\"ahrend \N{f\`i’u} und
\N{tsaw} die meisten Typen von Glieds\"atzen substantivieren k\"onnen, bevorzugen
die Verben des H\"orens, Sprechens und Fragens die Substantive \N{fmawn} f\"ur \D{Neuigkeigen},
\N{t\`i’eyng} f\"ur \D{Antwort} und \N{fayl\`i’u} f\"ur \D{diese W\"orter}. Es gibt weniger
Verschmelzungsm\"oglichkeiten:\label{morph:fmawn} 

\begin{center}
\begin{tabular}{rc}
Fall & Verschmelzung \\
\hline
Nominativ & \N{teynga} ($<$ \N{t\`i’eyng a}) \\
Agens & \N{teyngla} ($<$ \N{t\`i’eyng\`il a}) \\
Patiens & \N{teyngta} ($<$ \N{t\`i’eyngit a})
\end{tabular}
\end{center}
\index{fmawnta@\textbf{fmawnta}} \index{teynga@\textbf{teynga}}
\index{teyngla@\textbf{teyngla}} \index{teyngta@\textbf{teyngta}}
\index{fmawn@\textbf{fmawn}} \index{tiì'eyng@\textbf{t\`i’eyng}}
\index{fayluta@\textbf{fayluta}} \index{fayliì'u@\textbf{fayl\`i’u}}

\noindent F\"ur \N{fmawn} und \N{fayl\`i’u} gibt es Verschmelzungen nur f\"ur den
Patiens: \N{fmawnta} ($<$ \N{fmawnit a}) sowie \N{fayluta} ($<$ \N{fayl\`i’ut a}).
Siehe auch \horenref{syn:quot:nominalized} f\"ur den Satzbau.
\NTeri{31/8/2011}{http://naviteri.org/2011/08/reported-speech-reported-questions/}


\section{Das Adjektiv}
\subsection{Attribution} Attributive Adjektive sind mit ihrem Substantiv durch den
attributeven Affix \N{-a-} verbunden, welches auf der Seite ans Adjektiv geh\"angt
wird, die zum Substantiv zeigt, wie in \N{yerik awin} oder \N{wina yerik} f\"ur
"`ein schneller Yerik"'.\label{morph:adj-attr}
\index{-a-@\textbf{-a-}\index{Adjektiv!attributiv!Bildung}}

\subsubsection{} Ein mittels \N{le-} abgeleitetes Adjektiv verliert im Regelfalle
ein vorangestelltes (nicht aber ein angeh\"angtes) \N{a-}, also entweder
\N{ayftxoz\"a lefpom} oder \N{ayftxoz\"a alefpom}. Steht das Adjektiv jedoch dem
Substantiv voran, so beh\"alt es grunds\"atzlich das attributive \N{-a-}, wie in
\N{lefpoma ayftxoz\"a}.

\begin{center}
\begin{tabular}{ll}
\N{ayftxoz\"a lefpom} & der Regelfall \\
\N{ayftxoz\"a \uwave{a}lefpom} &  erlaubt \\
$*$\N{lefpom ayftxoz\"a} &  falsch \\
\N{lefpom\uwave{a} ayftxoz\"a} &  richtig \\
\end{tabular}
\end{center}


\section{Das Verb}
\subsection{Infixpositionen} Frommer weist drei Positionen f\"ur Verbinfixe aus:
Position 0, Position 1 und Position 2. Jede Position kann mit einer bestimmten
Art Infix besetzt werden (siehe unten).

\subsubsection{} Alle Infixe erscheinen in der letzten oder vorletzten Silbe des
Wortstamms und werden vor dem jeweiligen Vokal, Diphthong oder Pseudovokal
eingef\"ugt, beispielsweise wie in \N{k\"a} $>$ \N{k\INF{\`im}\"a} und
\N{taron} $>$ \N{t\INF{ol}ar\INF{ei}on}.

\subsubsection{} Wenn eine Silbe nicht \"uber einen einleitenden Konsonanten
verf\"ugt, so steht der Infix dennoch vor dem Vokal, so wie in \N{omum} $>$
\N{\INF{iv}omum} und \N{ftia} $>$ \N{fti\INF{ats}a}.

\subsubsection{} Die Betonung verbleibt auf dem Vokal, auf dem sie vor dem
Hinzuf\"ugen von Infixen bereits gelegen hatte: \N{\ACC{haw}nu} $>$
\N{h\INF{il\ACC{v}}\ACC{aw}nu}.\footnote{Ausnahme: Das Verb \N{o\ACC{mum}} verschiebt
die Betonung f\"ur alle konjugierten oder abgeleiteten Formen zum \N{o}:
\N{\INF{i\ACC{v}}\ACC{o}mum}, \N{n\`iaw\ACC{no}mum}.\index{omum@\textbf{omum}!Betonung}
Das Verb \N{i\ACC{nan}} folgt dem gleichen Muster: \N{\INF{o\ACC{l}}\ACC{i}nan}.
\index{inan@\textbf{inan}!Betonung}}

\subsubsection{} Im Regelfalle werden Infixe nur in einen Teil eines zusammengesetzten
Verbs eingef\"ugt. So ist \N{yomt\`ing}, \D{f\"uttern}, beispielsweise eine
Zusammensetzung aus \N{yom}, \D{essen}, und \N{t\`ing}, \D{geben}. Das Perfekt
hiervon ist nicht *\N{y\INF{ol}omt\`ing}, sondern \N{yomt\INF{ol}\`ing}.
In den meisten zusammengesetzten Verben steht das Verbelement am Schlu\ss{}, wohin
auch die Infixe gehen. In einigen wenigen zusammengesetzten Verben jedoch kommen die
Infixe ins erste Element, aber diese m\"ussen aus dem W\"orterbuch gelernt werden.
\index{Verb!zusammengesetztes!Infixe einf\"ugen}

\subsubsection{} In einer kleinen Anzahl Verben, die aus zwei Verben zusammengesetzt
worden sind, werden die Infixe auf beide Teile aufgeteilt, wie beispielsweise mit
\N{kan’ìn}, \D{sich spezielisieren in}, das aus \N{kan} \D{zielen, beabsichtigen} und
\N{’ìn} \D{besch\"aftigt sein} besteht.
\Ultxa{2/10/2010}{http://wiki.learnnavi.org/index.php/Canon/2010/UltxaAyharyu\%C3\%A4\%23Transitivity_and_Infix_Positions}

\subsection{Position 0} Diese Infixe ver\"andern die Transitivit\"at. Sie werden vor
dem Vokal der vorletzten Silbe, oder in einsilbigen Verben vor dem einzigen Vokal, eingef\"ugt.
\label{morph:pre-first}
\index{Reflexiv!Bildung}\index{Kausativ!Bildung}

\begin{center}
\begin{tabular}{lr}
Kausativ & \N{\INF{eyk}} \\
Reflexiv & \N{\INF{\"ap}} \\
\end{tabular}
\end{center}

\noindent\LNWiki{1/2/2010}{http://wiki.learnnavi.org/index.php/Canon\%23Reflexives_and_Naming} % reflexive
\LNWiki{15/2/2010}{http://wiki.learnnavi.org/index.php/Canon\%23Trials_.26_transitivity} % causative

\subsubsection{} Das kausative Reflexive, "`sich selbst veranlassen zu"' wird gebildet, indem
\N{\INF{\"ap}\INF{eyk}} eingef\"ugt wird. Daher ist \N{po t\"apeykerkup} \D{er veranla\ss{}t
sich selbst zu sterben}.\index{Reflexiv!eines Kausativs}

\subsection{Position 1} Diese bestimmen Zeitform, Aspekt und Stimmungslage und
erzeugen Partizipien. Sie werden vor dem Vokal der vorletzten Silbe, oder in einsilbigen
W\"ortern vor dem einzigen Vokal, eingef\"ugt. Sie folgen immer auf etwaige Infixe in
Position 0.

\begin{center}
\begin{tabular}{r|ccc}
 & Nur Zeitform & Perfektiv & Imperfektiv \\
\hline
Zukunft & \N{\INF{ay}, \INF{asy}} & \N{\INF{aly}} & \N{\INF{ary}} \\
Nahe Zukunft & \N{\INF{\`iy}, \INF{\`isy}} & \N{\INF{\`ily}} & \N{\INF{\`iry}} \\
allgemein    &  — & \N{\INF{ol}} & \N{\INF{er}} \\
Nahe Vergangenheit & \N{\INF{\`im}} & \N{\INF{\`ilm}} & \N{\INF{\`irm}} \\
Vergangenheit & \N{\INF{am}} & \N{\INF{alm}} & \N{\INF{arm}} \\
\end{tabular}
\end{center}
\LanguageLog{}
\LNWiki{27/1/2010}{http://wiki.learnnavi.org/index.php/Canon\%23Extracts_from_various_emails}
\LNWiki{19/2/2010}{http://wiki.learnnavi.org/index.php/Canon\%23Evidential} %ìrm

\subsubsection{} Die Zukunftsforment mit \N{s} zeigen ein Vorhaben an (\horenref{syn:verb:intenfut}).

\subsubsection{} Der subjunktive Infix \N{iv} verf\"ugt \"uber eingeschr\"ankte
Kombinationsm\"oglichkeiten mit weniger Abstufungsm\"oglichkeiten bei den Zeitformen.
\index{Subjunktiv!Infix}

\begin{center}
\begin{tabular}{r|ccc}
         & Nur Zeitform & Perfektiv & Imperfektiv \\
\hline
Zukunft & \N{\INF{\`iyev}, \INF{iyev}} & — & — \\
allgemein & \N{\INF{iv}} & \N{\INF{ilv}} & \N{\INF{irv}} \\
Vergangenheit & \N{\INF{imv}} & — & —
\end{tabular}
\end{center}

\noindent\LNWiki{9/1/2010}{http://wiki.learnnavi.org/index.php/Canon\%23Extracts_from_various_emails}
\LNWiki{30/1/2010}{http://wiki.learnnavi.org/index.php/Canon\%23Vocabulary_and_.3Ciyev.3E}
\LNWiki{30/1/2010}{http://wiki.learnnavi.org/index.php/Canon\%23Fused_-iv-_Infixes}

\subsubsection{} Es existieren lediglich zwei Partitipialinfixe. Diese lassen sich
nicht mit den Zeit-, Aspekt- oder Stimmungslageninfixen kombinieren.
\index{Partizip!Bildung}

\begin{center}
\begin{tabular}{lr}
Aktiv & \N{\INF{us}} \\
Passiv & \N{\INF{awn}} \\
\end{tabular}
\end{center}

\noindent Da Partizipien allesamt nicht-pr\"adikative Adjektive sind, werden sie
immer zusammen mit dem Attributivpartikel \N{-a-} verwendet
(\horenref{morph:adj-attr}, \horenref{syn:part:attr}).
\LNWiki{13/3/2011}{http://wiki.learnnavi.org/index.php/Canon/2010/March-June\%23Participial_Infixes}


\subsection{Position 2} Diese Infixe, die Gefallen oder Beurteilung durch den Sprecher
ausdr\"ucken, erscheinen in der letzten Silbe eines Verbs respektive hinter den Infixen
der ersten Position, wenn das Verb nur aus einer Silbe besteht.
\index{Einstellungsinfixe}\label{morph:verb:2nd-pos}

\begin{center}
\begin{tabular}{rl}
Positive Einstellung & \N{\INF{ei}}, \N{\INF{eiy}} (\horenref{l-and-s:eiy-epenth}) \\
Negative Einstellung & \N{\INF{\"ang}}, \N{\INF{eng}} (\horenref{l-and-s:eng}) \\
Formell, zeremoniell & \N{\INF{uy}} \\
schlu\ss{}folgernd, voraussetzend & \N{\INF{ats}} \\
\end{tabular}
\end{center}

\noindent\LNWiki{19/2/2010}{http://wiki.learnnavi.org/index.php/Canon\%23Evidential} % <ats>

\subsection{Beispiele} Die obigen Regeln sind etwas abstrakt, weshalb ich hier einige
Konjugationsbeispiele f\"ur einige Verbformen angebe.
Die Verben sind \N{eyk}, \D{(an)f\"uhren} als Beispiel f\"ur ein einsilbiges Wort ohne
einleitenden Konsonanten, \N{fpak}, \D{unterbrechen} f\"ur ein einsilbiges Wort mit einer
einleitenden Konsonantengruppe, \N{taron}, \D{jagen} als das zweisilbige Standardwort, das
Frommer in Beispielen verwendet und \N{yom·t\`ing} f\"ur ein zusammengesetztes Wort, bei
dem einzig dessen Ende konjugiert wird.

\begin{center}\small
\begin{tabular}{lllll}
           & \N{eyk} & \N{fpak} & \N{\ACC{ta}ron} & \N{\ACC{yom}·t\`ing} \\
\hline
Nahe Vergangenheit & \N{\`i\ACC{meyk}} & \N{fp\`i\ACC{mak}} & \N{t\`i\ACC{ma}ron} & \N{\ACC{yom}t\`im\`ing} \\
Reflexiv  & \N{\"a\ACC{peyk}} & \N{fp\"a\ACC{pak}} & \N{t\"a\ACC{pa}ron} & \N{\ACC{yom}t\"ap\`ing} \\
Refl., nahe Vergangenheit & \N{\"ap\`i\ACC{meyk}} & \N{fp\"ap\`i\ACC{mak}} & \N{t\"ap\`i\ACC{ma}ron} & \N{\ACC{yom}t\"ap\`im\`ing} \\
Zeremoniell & \N{u\ACC{yeyk}} & \N{fpu\ACC{yak}} & \N{\ACC{ta}ruyon} & \N{\ACC{yom}tuy\`ing} \\
Perf., zerem. & \N{olu\ACC{yeyk}} & \N{fpolu\ACC{yak}} & \N{to\ACC{la}ruyon} & \N{\ACC{yom}toluy\`ing} \\
Refl., perf., zerem. & \N{\"apolu\ACC{yeyk}} & \N{fp\"apolu\ACC{yak}} & \N{t\"apo\ACC{la}ruyon} & \N{\ACC{yom}t\"apoluy\`ing} \\
\end{tabular}
\end{center}

\noindent Die Bedeutungen einiger Beispiele dehnen die Sinnhaftigkeit bis zum
Brechen. Der Sinn ist einzig und alleine, die Infixpositionen anhand eines konsistenten
Satzes an Verbformen zu demonstrieren.