\nchapter{Buchstaben und Aussprache}


\section{System der Aussprache}
\noindent Das Na’vi verf\"ugt \"uber 20 Konsonanten, sieben Vokale und zwei
vokalische Resonanten, die Frommer als Pseudovokale bezeichnet.
\LanguageLog

\subsection{Konsonanten} 

\begin{center}
\begin{tabular}{llllll}
 & Labiale & Alveolare & Palatale & Velare & Glottale \\
Ejektive &	\N{px} [p'] & \N{tx} [t'] & & \N{kx} [k'] \\
Stimmlose Stops & \N{p} [p] & \N{t} [t] & & \N{k} [k] & \N{’} [ʔ] \\
Affrikate &             & \N{ts} {\gplus [t͡s]} \\
Stimmlose Reibelaute & \N{f} [f] & \N{s} [s] & & & \N{h} [h] \\
Stimmhafte Reibelaute & \N{v} [v] & \N{z} [z] \\
Nasale &         \N{m} [m] & \N{n} [n] & & \N{ng} [ŋ] \\
Flie\ss{}laute &         &  \N{r} {\gplus [ɾ]}, \N{l} [l] \\
Gleitlaute &       \N{w} [w] & &  \N{y} [j] \\
\end{tabular}
\end{center}

\subsubsection{} Stimmlose Stops am Anfang und in der Mitte eines Wortes
sind nicht gehaucht und an dessen Ende nicht freigegeben. Folgt jedoch
innerhalb eines Satzes ein Vokal auf einen Stopplaut, dann wird dieser
freigegeben, da die einzelnen W\"orter bei nat\"urlicher Aussprache ineinander
\"ubergehen. \N{oel se\uwave{t o}mum}.
Unterdr\"uckte Stopplaute fallen am ehesten bei gr\"o\ss{}eren Pausen auf,
so wie in \N{oel omum se\uwave{t}.}

\subsubsection{} Das \N{r} ist ein alveolarer Schlaglaut. Das \N{l}
entsteht klar im vorderen Bereich des Mundes, so wie in "`Laub"', im
Gegensatz zu dem velaren "`dunklen"' L aus dem Englischen, z. B. in
"`call"'.

\subsubsection{} Frommer hatte eine wissenschaftliche Rechtschreibung
entwickelt, die zwei der Digraphen als einzelne Buchstaben ausdr\"uckte,
\N{c} anstatt \N{ts} und \N{g} anstatt \N{ng}. Das System mit den Digraphen
war für die Schauspieler einfacher und wurde von Frommer ebenfalls in
Interviews in den Medien sowie den meisten seiner E-Mails verwendet.\label{l-and-s:cg}

\subsubsection{} Weil gew\"ohnliche Stopplaute als Silbenabschl\"usse
verwendet werden k\"onnen, ist die allgemeine Ejektivnotation \N{p’}
nicht eindeutig genug: \N{tsap’alute} ist nicht *\N{tsapxalute}.
\LNWiki{21/12/2009}{http://wiki.learnnavi.org/index.php/Canon\%23Extracts_from_various_emails}


\subsection{Vokale}

\begin{center}
\begin{tabular}{ccccc}
\N{i} [i], \N{\`i} [{\footnotesize I}]  & & & & \N{u} [u],[ʊ] \\
 & \N{e} {\gplus [ɛ]} & & & \N{o} [o] \\
 & & \N{\"a} [æ] &  \N{a} [a] \\
\end{tabular}
\end{center}

\subsubsection{} Das Phonem \N{u} ist in offenen Silben grunds\"atzlich [u]
und kann in geschlossenen Silben entweder [u] oder [ʊ] sein.
\LNWiki{20/5/2010}{http://wiki.learnnavi.org/index.php/Canon/2010/March-June\%23The_Dual_sounds_of_.22u.22}

\subsubsection{} Die Diphthonge sind \N{aw}, \N{ay}, \N{ew} und \N{ey}.
Lediglich in einem Diphthong kann ein \N{w} oder \N{y} am Silbenende
(\N{new}) oder vor einem Endkonsonanten (\N{hawng}) stehen.

\subsection{Pseudovokale}
Der Pseudovokal \N{rr} ist ein silbisches,
getrillertes {\gplus [r̩ː]} und \N{ll} ein silbisches {\gplus [l̩ː]}.

\subsection{Silbenstruktur}
Das Na’vi weist eine strenge, aber daf\"ur unkomplizierte Silbenstruktur auf.

\begin{itemize*}
  \item Eine Silbe mu\ss{} nicht zwingend mit einem Konsonanten beginnen
    (darf also mit einem Vokal beginnen).
  \item Eine Silbe mu\ss{} nicht zwingend auf einem Konsonanten enden
    (darf also mit einem Vokal aufh\"oren).
  \item Jeder Konsonant kann eine Silbe einleiten.
  \item Eine Konsonantengruppe bestehend aus \N{f, s, ts} $+$ \N{p, t, k,
    px, tx, kx, m, n, ng, r, l, w, y} kann eine Silbe einleiten (z. B.
    \N{tslam}, \N{ftu}).
  \item \N{Px tx kx p t k ’ m n l r ng} d\"urfen am Silbenende stehen.
  \item \N{Ts f s h v z w y} d\"urfen \textit{nicht} am Silbenende stehen.
  \item Am Silbenende gibt es keine Konsonantengruppen.
  \item \label{l-and-s:pseudo-no-null} Eine Silbe, die einen Pseudovokal
    enth\"alt, mu\ss{} mit einem Konsonanten oder einer Konsonantengruppe
    beginnen und darf keinen Endkonsonanten haben; dies spielt bei der
    Lenierung (\horenref{l-and-s:lenition:pseudovowel}) und der Deklination
    von Substantiven (\horenref{morph:decl:pseudovowel}) eine wichtige Rolle.
\end{itemize*}

\subsubsection{} Da eine Silbe nicht notwendigerweise über einen einleitenden oder
abschlie\ss{}enden Konsonanten verf\"ugen mu\ss{}, ist es nicht un\"ublich, da\ss{}
mehrere Vokale in einem Wort nebeneinander stehen. In diesem Fall ist jeder Vokal
eine eigene Silbe. \N{mui\"a} [mu.i.æ], \N{ioang} [i.o.aŋ].

\subsubsection{} Allgemein wird die Abfolge VKV als V.KV in Silben getrennt
statt VK.V, daher wird \N{tsenge} zu [tsɛ.ŋɛ] statt zu *[tsɛŋ.ɛ].
Lautmalerei kann dies aufheben, wie z. B. in \N{kxang\-ang\-ang} [k'aŋ.aŋ.aŋ],
da hier der Echoeffekt erwünscht ist.

\subsubsection{} Im Na’vi gibt es keine langen Vokale, d. h. identische
Vokale stehen niemals nebeneinander (siehe dazu \horenref{l-and-s:contract}).

\subsubsection{} Doppelte Konsonanten treten in Stammw\"ortern nicht auf,
k\"onnen jedoch an Morphemgrenzen entstehen, beispielsweise in Ableitungen wie
\N{tsukk\"ateng} $<$ \N{tsuk-} $+$ \N{k\"ateng} oder aber in Enkliktikonen wie
\N{Mo’atta} $<$ \N{Mo’at} $+$ \N{ta} (\horenref{l-and-s:stress:enclisis}).
% http://naviteri.org/2011/03/“receptive-ability”-and-hesitation/comment-page-1/#comment-604

\subsubsection{} Wie in den meisten menschlichen Sprachen auch brechen
manche Interjektionen diese Regeln, so auch \N{o\`isss}, ein Laut des \"Argers,
oder auch \N{saa}, ein Drohruf.


\subsection{Betonung}
Jedes Wort des Na’vi verf\"ugt \"uber mindestens eine Betonung, die sich
jedoch nicht vorhersagen l\"a\ss{}t. In einigen sehr wenigen Situationen
unterscheiden sich W\"orter, die identisch geschrieben werden, nur durch
ihre Betonung, beispielsweise \N{\ACC{tu}te}, \D{Person} im Gegensatz zu
\N{tu\ACC{te}}, \D{Frau}.

\subsubsection{} Einzig f\"ur dieses Wort, \D{Frau}, darf im Na’vi ein
Akzent geschrieben werden, um die Betonung anzuzeigen: \N{tut\'e}.
\index{tuté@\textbf{tut\'e}}

\subsubsection{} Manche Wortbildungsprozesse k\"onnen Verschiebungen in
der Betonung verursachen (\horenref{lingop:prefix:ke}, \horenref{lingop:suffix:gender}).

\subsubsection{} S\"amtliche Adpositionen sowie einige wenige Konjunktionen
und Partikel und k\"onnen enkliktisch werden. Dabei verlieren sie ihre eigene
Betonung und werden effektiv zu einem Teil des Wortes, an das sie geh\"angt
werden, und werden entsprechend geschrieben.
\N{\ACC{tsa}ne} ($<$ \N{tsaw} $+$ \N{ne}), \N{ho\ACC{ren}\-ti\-si}
($<$ \N{ho\ACC{ren}ti} $+$ \N{si}).
\label{l-and-s:stress:enclisis}\index{Enkliktik}

\subsubsection{} Obwohl ein zusammengesetztes Substantiv als ein Wort
geschrieben wird, behalten die Teile ihre urspr\"ungliche Betonung, wie in
\N{ti\ACC{re}a\ACC{fya}’o}, \D{Pfad des Geistes}.
\index{Zusammengesetztes Wort!Betonung} 

%\subsubsection{} Word stress is a property of stem words.  No matter
%how many affixes a root word takes, no secondary accents develop.

\subsection{Gesprochenes Alphabet}
Au\ss{}er f\"ur den \N{t\`iftang}, den Glottisstop, enthalten die Bezeichner
der einzelnen Phoneme Informationen dar\"uber, wie sie eingesetzt werden.
Zudem weisen sie eine ungew\"ohnliche Gro\ss{}schreibung auf, wenn sie
ausgeschrieben werden: \index{Alphabet!gesprochen}

\begin{center}\small
\begin{tabular}{lll}
\N{T\`iftang} & \N{\`I} & \N{ReR} \\
\N{A}  & \N{KeK}   & \N{’Rr} \\
\N{AW} & \N{KxeKx} & \N{S\"a} \\
\N{AY} & \N{LeL}   & \N{TeT} \\
\N{\"A}  & \N{’Ll}   & \N{TxeTx} \\
\N{E}  & \N{MeM}   & \N{Ts\"a} \\
\N{EW} & \N{NeN}   & \N{U} \\
\N{EY} & \N{NgeNg} & \N{V\"a} \\
\N{F\"a} & \N{O}     & \N{W\"a} \\
\N{H\"a} & \N{PeP}   & \N{Y\"a} \\
\N{I}  & \N{PxePx} & \N{Z\"a} \\
\end{tabular}
\end{center}

\subsubsection{} Vokale und Diphthonge werden einfach als sie selbst
ausgesprochen und geschrieben. Den Pseudovokalen wird ein Glottisstop
vorangestellt, da sie einen einleitenden Konsonanten ben\"otigen (\horenref{l-and-s:pseudo-no-null}).

\subsubsection{} Die Bezeichner f\"ur diejenigen Konsonanten, die nicht am
Ende einer Silbe stehen d\"urfen, werden durch Anh\"angen eines \N{\"a} gebildet,
wie beispielsweise in \N{Ts\"a}. Diejenigen, die am Ende einer Silbe stehen
d\"urfen, verwenden den Vokal \N{e} und wiederholen den Konsonanten am Ende
des Bezeichners, wie beispielsweise in \N{PeP}.


\section{Lenition}
\noindent Bestimmte grammatikalische Prozesse verursachen Ver\"anderungen im
ersten Konsonanten eines Wortes. Diese Ver\"anderung wird "`Lenition"'
genannt. Lediglich acht Konsonanten k\"onnen leniert werden. \index{Lenition}\label{l-and-s:lenition}
\LanguageLog

\begin{center}
\begin{tabular}{lll}
Konsonant & Lenition & Beispiel \\
\N{px, tx, kx} & \N{p, t, k} & \N{\uwave{tx}ep} wird zu \N{m\`i \uwave{t}ep} \\
\N{p, t, k} & \N{f, s, h} & \N{\uwave{k}elku} wird zu \N{ro \uwave{h}elku} \\
\N{ts} & \N{s} & \N{\uwave{ts}mukan} wird zu \N{ay\uwave{s}mukan} \\
\N{’} & verschwindet & \N{’eylan} wird zu \N{fpi eylan} \\
\end{tabular}
\end{center}

\subsection{Glottisstop} Der Glottisstop wird nicht leniert, wenn er einem
Pseudovokal vorangestellt ist (\N{m\`i ’Rrta} anstatt *\N{m\`i Rrta}).
\index{Glottisstop!Lenition}\label{l-and-s:lenition:pseudovowel}

\subsection{Adpositionen} Einige wenige Adpositionen verursachen Lenition, wenn
sie einem Wort vorangestellt werden: \N{fpi}, \N{\`il\"a}, \N{m\`i}, \N{nu\"a}, \N{ro},
\N{sre} (davon abgeleitet \N{lisre} und \N{pxisre}), \N{w\"a}. Werden sie an ein
Wort angeh\"angt, dann wird keine Lenition verursacht, weder in dem Wort, an das
sie geh\"angt worden sind, noch in dem folgenden Wort.
\index{fpi@\textbf{fpi}!Lenition}\index{ilä@\textbf{\`il\"a}!Lenition}
\index{miì@\textbf{m\`i}!Lenition}\index{ro@\textbf{ro}!Lenition}
\index{sre@\textbf{sre}!Lenition}\index{pxisre@\textbf{pxisre}!Lenition}
\index{waä@\textbf{w\"a}!Lenition}\index{nuaä@\textbf{nu\"a}!Lenition}
\index{Lenition!Adpositionen}\index{Adpositionen!Lenition}

\subsection{Numeruspr\"afixe} Pr\"afixe, die Lenition verursachen, werden mit
einem Pluszeichen statt einem Bindestrich gekennzeichnet, wie in \N{ay+}, dem
lenisierenden Pluralpr\"afix.\index{Lenition!Numeruspr\"afixe}

\subsection{Fragepronomen} Wird das Fragepronomen \N{pe+} als Pr\"afix verwendet,
dann verursacht es Lenition (\horenref{morph:pre:pe}).

\subsection{Zahlen} Wenn als Suffix verwendet, werden die abh\"angigen Formen
der Zahlen lenisiert (\horenref{numbers:dependent}). \index{Lenition!Zahlen}


\section{Morphophonologie}

\subsection{Vokalverschmelzungen} Da identische Vokale nicht nebeneinander
stehen d\"urfen, beinhalten einige grammatikalische Prozesse die Verschmelzung
eines doppelten Vokals zu einem einzigen.\index{Vokal!Verschmelzung}\label{l-and-s:contract}

\subsubsection{} Das Adjektivmorphem \N{-a-} verschwindet, wenn es an
ein einleitendes oder abschlie\ss{}endes \N{a} eines Adjektivs geh\"angt wird,
wie in \N{apxa tute} anstatt *\N{apxaa tute}.
\index{-a-@\textbf{-a-}!mit \textbf{a} in einem Adjektiv}
\index{Adjektiv!Verschmelzung}

\subsubsection{} Hinterlassen die Dual- und Trialpr\"afixe eine Kette aus zwei
\N{e}, wie in \N{me} $+$ \N{’eveng} $>$ *\N{meeveng} (beachten Sie die Lenition),
dann verschmelzen die beiden Vokale zu einem: \N{meveng}.
\label{l-and-s:phonotactics:nsc} \index{Dual!Verschmelzung}
\index{Trial!Verschmelzung}
\LNWiki{20/1/2010}{http://wiki.learnnavi.org/index.php/Canon\%23Extracts_from_various_emails}

\subsubsection{} Enden Pronominalpr\"afixe auf dem gleichen Buchstaben,
mit dem das folgende Wort beginnt, verschmelzen diese zu einem, wie
in \N{tsatan} $<$ \N{tsa-} $+$ \N{atan}, \N{f\`lva} $<$ \N{f\`i-} $+$ \N{\`ilva}
(\horenref{morph:prenoun:contraction}).\footnote{Der Glottisstop z\"ahlt
als Konsonant, daher entsteht \N{f\`i’\`iheyu} aus \N{f\`i-} $+$ \N{’\`iheyu}.}
\label{l-and-s:phonotactics:precontract}\index{Pronomen!Verschmelzung}
\LNWiki{18/5/2011}{http://wiki.learnnavi.org/index.php/Canon/2011/April-December\%23Kawtseng.2C_tsapo_and_prefixes}

\subsubsection{} Eine Verschmelzung tritt nicht auf, wenn das indefinite \N{-o}
respektive enkliktisce Adpositionen verwendet werden. Stehen zwei identische Vokale
nebeneinander, dann werden diese durch einen Bindestrich getrennt, siehe \N{fya’o-o},
\D{irgendein Weg} oder \N{zekw\"a-\"ao}, \D{unter irgendeinem Finger}.\footnote{
Obwohl das Na’vi praktisch keine langen Vokale kennt, kommt dieser Effekt in
dieser Situation zum Tragen. Achten Sie daher darauf, da\ss{} Sie beide \N{\"a}
in einem Wort wie \N{zekw\"a-\"ao} betonen.}\index{Vokal!Verschmelzung!unterdr\"uckt}
% http://wiki.learnnavi.org/index.php/Canon/2010/UltxaAyharyuä#Phonological_Questions

\subsection{Verschmelzung von Pseudovokalen} Aufgrund der Gestaltung der
Aspektinfixe \N{\INF{er}} und \N{\INF{ol}} ist es m\"oglich, da\ss{} die
Pseudovokale direkt hinter ihrem konsonantischen Gegenst\"uck stehen k\"onnen,
beispielsweise in *\N{p\INF{ol}ll\ACC{txe}}. Geschieht dies in einer
unbetonten Silbe, verschwindet der Pseudovokal (\N{pol\ACC{txe}}). In einer
betonten Variable, verschwindet der Infix (*\N{\ACC{f}\INF{er}\ACC{rr}fen} $>$
\N{\ACC{frr}fen}.\index{Pseudovokal!Verschmelzung}
\LNWiki{23/3/2010}{http://wiki.learnnavi.org/index.php/Canon/2010/March-June\%23Misc_Answers}

\subsection{Lauteinf\"ugung bei Affektinfixen} Folgt dem positiven Infix
\N{\INF{ei}} der Vokal \N{i} oder\N{\`i}, wird ein \N{y} eingef\"ugt, beispielsweise
\N{seiyi} $<$ *\N{s\INF{ei}i} oder \N{veykrreiy\`in} $<$ *\N{veykrr\INF{ei}\`in}\label{l-and-s:eiy-epenth}

\subsection{Assimilation von Nasalen} In vielen zusammengesetzten W\"ortern und
einigen Redewendungen werden Nasale der Position des folgenden Wortes angepa\ss{}t,
wie in \N{lumpe}, einer Variation von \N{pelun}. Solche Anpassungen werden nicht
immer ausgeschrieben, was die Etymologie eines Wortes klarer werden l\"a\ss{}t,
beispielsweise in \N{zenke} statt *\N{zengke}, abgeleitet von \N{zene ke} oder
verschiedenen Redewendungen mit dem Verb \N{t\`ing}, \D{geben}. So wird \N{t\`ing m\`ikyun}
als \N{t\`im m\`ikyun} ausgesprochen.\index{Assimilation von Nasalen}\label{l-and-s:nasalassim}

\subsection{Harmonie von Vokalen} Das Na’vi kennt zwei F\"alle optionaler
regressiver Harmonie von Vokalen in Verbinfixen.\index{Vokal!Harmonie}

\subsubsection{} Der subjunktive Infix f\"ur das Futur, \N{\INF{iyev}}, erscheint
am h\"aufigsten als \N{\INF{\`iyev}} mit einer Verst\"arkung des ersten Vokals.

\subsubsection{} Der Vokal des negativen Infixes \N{\INF{\"ang}} kann aufgehellt
werden, wenn ihm direkt der Vokal \N{i} folgt, woraufhin dieses zu \N{\INF{eng}} wird,
wie in \N{tsap’alute \uwave{sengi} oe}.\label{l-and-s:eng} 
\Ultxa{2/10/2010}{http://wiki.learnnavi.org/index.php/Canon/2010/UltxaAyharyu\%C3\%A4\%23.C3.A4ng.2Feng}

\subsection{Elision} In schneller Aussprache wird ein \N{-e} am Wortende
fallengelassen, wenn das folgende Wort mit einem Vokal anf\"angt. \Npawl{K\`iyevam$\not$e
ult$\not$e Eywa ngahu}. Dies wird beim Schreiben nicht angezeigt.\index{Elision}
\QUAESTIO{Aber nicht bei einsilbigen W\"ortern? \N{ke}? \N{sre}?}

\subsubsection{} Der Vokal \N{\`i} in \N{m\`i-}, \N{s\`i-} und dem Adverbpr\"afix \N{n\`i-}
entf\"allt vor dem Pluralpr\"afix \N{ay+}, obwohl sich beim Schreiben nichts
\"andert. So wird \N{n\`iayfo}, \D{wie sie} als \N{nayfo} ausgesprochen.\label{l-and-s:elision-i} 
\index{miì@\textbf{m\`i}!Elision mit Plural}
\index{siì@\textbf{s\`i}!Elision mit Plural}
\index{niì-@\textbf{n\`i-}!Elision mit Plural}
\NTeri{1/7/2010}{http://naviteri.org/2010/07/thoughts-on-ambiguity/}



\section{Orthographische Konventionen}
\noindent Das Na’vi folgt im Allgemeinen den Schreib-, Zeichensetzungs- und
Gro\ss{}schreibregeln des Englischen, aber es gibt einige Unterschiede.

\subsection{Eigennamen} Eigennamen behalten ihre urspr\"ungliche Gro\ss{}schreibung,
wenn ihnen lexikalische Pr\"afixe vorangestellt werden, wie in \N{l\`i’fya le\uwave{Na’vi}}.

\subsection{Zitate} Direkte Zitate werden im Na’vi nicht in Anf\"uhrungszeichen
gesetzt. Stattdessen werden die Zitatpartikel \N{san\dots s\`ik} (siehe \horenref{syn:direct-quote}).
\index{Zitate!Zeichensetzung}

\subsection{Etymologische Schreibweise} Zus\"atzlich zur gelegentlichen Schreibweise
von Nasalen, die die Etymologie widerspiegelt (\horenref{l-and-s:nasalassim})
gibt es einige wenige grammatikalische Prozesse, die eine Schreibweise zur
Folge haben, die die Grammatik st\"arker repr\"asentiert als die Betonung.

\subsubsection{} Die Wurzel f\"ur das Pronomen der ersten Person, \N{oe}, beh\"alt,
obwohl es \N{we} betont wird, wenn ein Suffix angeh\"angt wird, seine urspr\"ungliche
Schreibweise. (\horenref{morph:pron:oe-we}).

\subsubsection{} Vor W\"ortern, die mit \N{y} anfangen, bleibt der Pluralpr\"afix \N{ay+}
unver\"andert, wie in \N{ayyerik}.
\LNWiki{18/4/2010}{http://wiki.learnnavi.org/index.php/Canon/2010/March-June\%23ay.2Byerik}

\subsection{Kopplung von Satzgliedern} Bestimmte kurze Attributivs\"atze werden
mit Bindstrichen geschrieben, um die Elemente zu verbinden.

\subsubsection{} Satzglieder f\"ur Farben, die das \N{na}, \D{so wie}, verwenden,
werden mit Bindestrichen versehen. \N{f\`isyulang aean-na-ta’leng}, \D{diese hautblaue Blume}
(\horenref{syn:attr:na}).

\subsubsection{} Partizipien von mit \N{si} gebileten Verben werden ebenfalls mit
Bindestrichen versehen. \N{srung-susia tute}, \D{eine helfende Person}
(\horenref{syn:participle:si-const}).