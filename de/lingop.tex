\nchapter{Wortaufbau}

\section{Ableitende Affixe}\index{Affix!ableitend}
\noindent Das Na'vi verf\"ugt \"uber eine Reihe Affixe, die zum Erzeugen neuer
W\"orter eingesetzt werden k\"onnen. Einige \"andern lediglich die Wortgruppe und
erzeugen so z. B. aus einem Substantiv ein Adjektiv. Dennoch sollten Sie nicht
davon ausgehen, da\ss{} diese Affixe vollst\"andig produktiv und die Bedeutungen
der abgeleiteten W\"orter nicht hundertprozentig vorhersagbar sind.\label{lingop:affixes}
Sie k\"onnen sich nur unter Zuhilfenahme eines W\"orterbuchs ob der Bedeutung eines
abgeleiteten Wortes sicher sein (siehe auch \horenref{syn:nifyao} f\"ur Adverben).
Die unten angef\"uhrten Affixe sind nicht vollst\"andig produktiv, wenn nicht anders
angegeben.

\medskip
\noindent Obwohl es aussagekr\"aftige Muster gibt, wie sich die Betonung durch
bestimmte Ableitungsprozesse ver\"andert, gibt es daf\"ur keine Regeln, die
keinerlei Ausnahmen beinhalten. Auch was die Betonung abgeleiteter W\"orter betrifft,
k\"onnen Sie sich nur unter Zuhilfenahme eines W\"orterbuchs vollst\"andig sicher sein.


\subsection{Pr\"afixe} Diese ableitenden Pr\"afixe verschieben die Betonung nur
selten auf eine andere Silbe. \N{ngay} $>$ \N{t\`i\ACC{ngay}}.

\subsubsection{} \N{Le-} leitet Adjektive von Substantiven ab, wie
\N{le\ACC{hrr}ap} \D{gef\"ahrlich} (von \N{\ACC{hrr}ap}, \D{Gefahr}).
\index{le-@\textbf{le-}}

\subsubsection{} \N{N\`i-} leitet Adverben von Substantiven, Pronomen, Adjektiven und
Verben ab, beispielsweise \N{n\`iNa’vi}, \D{"`na’viisch"', auf Na’vi} von
\N{Na'vi}, \N{n\`iayfo}, \D{wie sie}, \N{n\`i\ACC{ftu}e}, \D{einfach} von
\N{\ACC{ftu}e} \D{einfach} und \N{n\`i\ACC{tam}}, \D{genug} von \N{tam}
\D{gen\"ugen}.\index{niì-@\textbf{n\`i-}}

\subsubsection{} \N{S\"a-} leitet instrumentale Substantive von Verben und Adjektiven ab,
wie in \N{s\"a\ACC{nu}me} \D{Lehre, Unterweisung} von \N{\ACC{nu}me} und
\N{s\"a\ACC{spxin}}, \D{Krankheit} von \N{spxin}, \D{krank}.\index{saä-@\textbf{s\"a-}}

\subsubsection{} \N{T\`i-} leitet Substantive von Adjektiven, Verben und gelegentlich auch
anderen Substantiven ab, wie in \N{t\`i\ACC{ngay}}, \D{Wahrheit} von
\N{ngay}, \D{wahr}, \N{t\`ifti\ACC{a}}, \D{Studien} von
\N{fti\ACC{a}}, \D{studieren}, \N{t\`i\ACC{’awm}} \D{Lagern (Subst.)} von \N{’awm}
\D{Lager}.\index{tiì-@\textbf{t\`i-}}

\subsubsection{} Stammsilben k\"onnen durch diese Pr\"afixe einen Vokal verlieren, wenn
der einleitende Konsonant auch ein g\"ultiger Abschlu\ss{} ist: \N{n\`im\ACC{wey}pey}
\D{geduldig} $<$ \N{ma\ACC{wey}·pey} \D{geduldig sein}.


\subsection{Verneinender Pr\"afix} Einige W\"orter, im Regelfalle, aber nicht ausschlie\ss{}lich,
Adjektive, werden erzeugt, indem das Wort \N{ke}, also \D{nicht}, als Pr\"afix vorangestellt wird.

\subsubsection{} Steht \N{ke-} vor dem adjektivischen Pr\"afix \N{le-}, so wird Letzterer
zu einem \N{-l-} reduziert, wie in \N{kel\ACC{tsun}}, \D{unm\"oglich} im Vergleich zu
\N{le\ACC{tsun}slu}, \D{m\"oglich}, sowie \N{kel\ACC{fpom}tokx}, \D{ungesund} von
\N{lefpom\ACC{tokx}} \D{gesund}.

\subsubsection{} Steht \N{le-} vor dem verneinenden Pr\"afix, so wird Letzterer zu einem
\N{-k-} reduziert, wie in \N{lek\ACC{ye}’ung}, \D{irre} von \N{ke\ACC{ye}’ung} \D{Irrsinn}.

\subsubsection{} Der Pr\"afix \N{ke-} kann auf Stammadjektive und Partizipien, wodurch die
Betonung sich auf das \N{ke-} verschiebt, wie in \N{\ACC{ke}teng}, \D{unterschiedlich} von
\N{teng} \D{gleich, identisch} und \N{\ACC{ke}rusey}, \D{tot} von
\N{ru\ACC{sey}}, \D{lebend}. Beachten Sie aber: \N{key\ACC{awr}}, \D{falsch} von
\N{ey\ACC{awr}}, \D{richtig}.\label{lingop:prefix:ke}

\subsubsection{} Der Pr\"afix \N{ke-} kann auch Substantive erzeugen und sich mit diesen
kombinieren, wie beispielsweise \N{ke\ACC{ye}’ung}, \D{Irrsinn}, sowie \N{\ACC{ke}tuwong},
\D{Fremdling, Au\ss{}erirdischer}.
Es gibt zu wenig Beispiele, um das Verhalten der Betonung bestimmen zu k\"onnen.


\subsection{Adverbiales "`a-"'} Zwei Zustandsverben, \N{l\`im} \D{fern sein} und
\N{sim} \D{nah sein} haben die adverbialen Formen \N{a\ACC{l\`im}}, \D{entfernt} und
\N{a\ACC{sim}} \D{nahe}. Diese werden als versteinerte Abk\"urzungen von Formen wie
\N{nìfya'o a l\`im} (\horenref{syn:nifyao}). Hierbei handelt es sich um feststehende
lexikalische Einheiten, die \"uber keine Formen wie *\N{l\`ima} and *\N{sima} verf\"ugen.
\index{-a-@\textbf{-a-}!mit Adverben}\index{aliìm@\textbf{al\`im}}\index{asim@\textbf{asim}}
\LNWiki{17/5/2010}{http://wiki.learnnavi.org/index.php/Canon/2010/March-June\%23Near.2C_Distant_and_Irregular_Adverbs}


\subsection{Pr\"afix und Infix} Es gibt eine einzige Ableitung, die sowohl einen Pr\"afix
als auch einen Infix verwendet.

\subsubsection{} \N{T\`i- \INF{us}} erzeugt ein Gerundium. Es ist f\"ur Verbwurzeln sowie
zusammengesetzte Verben vollst\"andig produktiv (von \N{si}-konstruierte Verben k\"onnen
keine Gerundien abgeleitet werden). Dies ist insbesondere dann hilfreich, wenn eine
einfache Ableitung mit \N{t\`i-} bereits \"uber eine eigene Bedeutung verf\"ugt, wie
\N{rey}, \D{leben (Vin.)}, also \N{t\`i\ACC{rey}}, \D{Leben (Subst.)}, aber
\N{t\`iru\ACC{sey}} \D{lebendig}. Siehe auch \horenref{syn:gerund}.
\QUAESTIO{Wie sieht es mit \N{yomt\`ing} aus? \N{Yomt\`itus\`ing}?} % May 2011
\index{Gerundium!Bildung}\label{lingop:gerund}
\index{si-Konstruktion@\textbf{si}-Konstruktion!Kein Gerundium}



\subsection{Mittlersuffixe} Diese Suffixe haben ebenfalls keine Verschiebung der Betonung zur Folge.

\subsubsection{} \N{-tu} erzeugt Substantive aus W\"ortern, die keine Verben sind, wie
\N{\ACC{pam}tseotu}, \D{Musiker} von \N{\ACC{pam}tseo} \D{Musik} oder \N{tsul\ACC{f\"a}tu},
\D{Meister, Experte} von \N{tsul\ACC{f\"a}} \D{K\"onnen (Subst.)}.\index{-tu@\textbf{-tu}}

\subsubsection{} \N{-yu} erzeugt Substantive aus Verben, wie in \N{taronyu}, \D{J\"ager} von
\N{taron}, \D{jagen}. Dieser Suffix ist vollst\"andig produktiv.\index{-yu@\textbf{-yu}}
\NTeri{11/7/2010}{http://naviteri.org/2010/07/diminutives-conversational-expressions/}% -yu productivity



\subsection{Diminutivsuffix} Der unbetonte Suffix \N{-tsy\`ip} kann ohne Einschr\"ankungen
eingesetzt werden, um Verkleinerungen aus Substantiven und Verben zu erhalten. Eigennamen
k\"onnen Silben verlieren, wenn dieser Suffix angeh\"angt wird: \N{Kamtsy\`ip} oder
\N{Kamuntsy\`ip} f\"ur \D{kleiner Kamun}. Die Verkleinerungsform hat drei Einsatzgebiete.
\label{lingop:dimin}\index{Diminutiv}\index{tsyiìp@\textbf{-tsy\`ip}}
\NTeri{11/7/2010}{http://naviteri.org/2010/07/diminutives-conversational-expressions/}

\subsubsection{} Zuerst einmal kann der Diminutiv eine haupts\"achlich lexikalische
Ableitung sein. Solche W\"orter finden sich im W\"orterbuch wieder, wie \N{puktsy\`ip},
also \D{Brosch\"ure, Flugblatt, Pamphlet}, abgeleitet von \N{puk}, \D{Buch}. Die
St\"arke der Verkleinerung f\"allt schwach genug aus, so da\ss{} man das Adjektiv
\N{tsawl}, \D{gro\ss{}} zusammen mit dem Diminutiv verwenden kann, ohne einen Widerspruch
hervorzurufen, beispielsweise \N{tsawla utraltsy\`ip}, \D{gro\ss{}er Busch}.

\subsubsection{} Zweitens kann der Diminutiv Gefallen und Z\"artlichkeit ausdr\"ucken.
\Npawl{Za’u f\`i\-tseng, ma ’itetsy\`ip}, \D{Komm her, kleine Tochter}.
Diese Verwendung sollte nicht so aufgefa\ss{}t werden, als impliziere sie
ein Alter. Die Tochter aus dem Beispielsatz k\"onnte sehr wohl eine Erwachsene
sein.

\subsubsection{} Drittens kann der Diminutiv verwendet werden, um Verunglimpfung und
Beleidigungen auszudr\"ucken.\Npawl{F\`itaron\-yu\-tsy\`ip ke tsun ke’ut stivä’n\`i},
\D{Dieses (wertlose) J\"agerlein kann \"uberhaupt nichts fangen}. Der verunglimpfende
Tonfall kann auch gegen einen selbst gerichtet sein.
\Npawl{Nga n\`iawnomum to \uwave{oetsy\`ip} lu txur n\`itxan}, \D{Wie jeder wei\ss{}
bist Du viel st\"arker als \uwave{meine alte Wenigkeit}}. Man kann den verunglimpfenden
vom herzlichen Gebrauch nur anhand des Kontextes unterscheiden.



\subsection{Geschlechtsspezifische Suffixe} Die geschlechtsspezifischen Suffixe sind
dahingehend ungew\"ohnlich, weil sie nicht nur auf Substantive angewendet werden k\"onnen,
sondern auch auf das Pronomen der dritten Person (\horenref{morph:pron:gender}).
\label{lingop:suffix:gender}

\subsubsection{} Der Suffix \N{-an} steht f\"ur M\"anner, wie in \N{po\ACC{an}}, \D{er}
und \N{\ACC{’i}tan}, \D{Sohn}.

\subsubsection{} Der Suffix \N{-e} steht f\"ur Frauen, wie in \N{po\ACC{e}}, \D{sie}
und \N{\ACC{’i}te}, \D{Tochter}.

\subsubsection{} Der Einflu\ss{} dieser Suffixe auf die Betonung l\"a\ss{}t sich nicht
vorherbestimmen. \N{tu\ACC{tan}}, \D{Mann} von \N{\ACC{tu}te}
\D{Person}, aber \N{mun\ACC{txa}tan}, \D{Ehemann, Gef\"ahrte (m.)} von
\N{mun\ACC{txa}tu}, \D{Ehegatte, Gef\"ahrte}.


\section{Verdoppelung}

\subsection{Iteration} \QUAESTIO{K\"onnen \N{letrrtrr} und \N{krro krro} verallgemeinert
werden?}



\section{Zusammengesetzte W\"orter}

\subsection{Ausrichtung} Das dominante Element eines zusammengesetzten Wortes kann
im Na’vi an erster oder letzter Stelle stehen.\footnote{Viele menschliche Sprachen
sind da wesentlich restriktiver. Das dominante Element eines zusammengesetzten Wortes
im Deutschen, das auch als "`Kopf"' bezeichnet wird, steht im Regelfalle hinten, wie
in \textit{Fliegenpilz}, \textit{Taschenlampe} oder \textit{Schiefertafel}. Im
Gegensatz dazu stellt das Vietnamesische das Kopfelement f\"ur eigene zusammengesetzte
W\"orter nach vorn und f\"ur aus dem Chinesischen importierte W\"orter nach hinten.}
Jedoch gibt es eine starke Tendenz zu Zusammensetzungen, deren Kopfelement am Ende
steht. Zusammengesetzte Verben haben das Kopfelement am h\"aufigsten vorne.

\subsubsection{} Die Wortgruppe eines zusammengesetzten Wortes entspricht der
Gruppe, zu der das Kopfelement geh\"ort, von daher ist \N{txam\ACC{pay}},
\D{See} ein Substantiv, weil \N{pay}, \D{Wasser} ein Substantiv ist.

\subsubsection{} Wie Stammw\"orter auch k\"onnen zusammengesetzte W\"orter ihre
Gruppenzugeh\"origkeit \"andern, wenn die oben genannten ableitenden Affixe
verwendet werden. \N{lefpom\ACC{tokx}}, \D{gesund} von \N{fpom\ACC{tokx}}, \D{Gesundheit}.


\subsection{Endsilbenschwund} W\"orter k\"onnen Teile verlieren, wenn sie in einem
zusammengesetzten Wort verwendet werden: \N{\ACC{ven}zek}, \D{Zeh} $<$ \N{\ACC{ve}nu},
\D{foot} $+$ \N{\ACC{zek}w\"a}, \D{Finger} und \N{s\`il\ACC{pey}}, \D{Hoffnung} $<$
\N{s\`iltsan}, \D{gut} $+$ \N{pey}, \D{warten (auf)}.


\subsection{Konstruktionen mit "`si"'} Der g\"angige Weg, um ein Substantiv oder Adjektiv
in ein Verb zu wandeln, ist, dem nicht deklinierten Substantiv oder Adjektiv das abstrakte Verb
\N{si} nachzustellen, welches nur in solchen Konstruktionen Anwendung findet. Die Reihenfolge
steht fest als \N{(Subst.) si}, wobei das \N{si} alle Verbaffixe bekommt.\label{lingop:si-const}
\index{si-Konstruktion@\textbf{si}-Konstruktion}

\subsubsection{} Das Verb \N{irayo si} hat eine nicht so strikte Reihenfolge.
\LNWiki{12/5/2010}{http://wiki.learnnavi.org/index.php/Canon/2010/March-June\%23Word_Order_Issues}

\subsubsection{} Verneinungen durchbrechen die normale Abfolge von \N{(Subst.) si}. \N{Oe pamrel
ke si}, \D{Ich schreibe nicht} (\horenref{syn:neg:si-const}), \N{Txopu r\"a’\"a si} \D{Hab keine
Angst} (\horenref{syntax:prohibitions}).


\section{H\"aufige und erw\"ahnenswerte Wortkomponenten}

\subsection{-fkeyk} Dieser unbetonte Suffix, der von \N{t\`ifkeytok}, \D{Zustand, Situation}
abgeleitet wird, erzeugt einige W\"orter mit speziellen, idiomatischen Bedeutungen,
beispielsweise \N{\ACC{ya}fkeyk}, \D{Wetter}. Dennoch ist es weitestgehend produktiv.
(\Npawl{Kilvanfkeyk lu fyape f\`itrr?} \D{Wie ist der Zustand des Flusses heute?}
\index{-fkeyk@\textbf{-fkeyk}}
\NTeri{1/4/2011}{http://naviteri.org/2011/04/yafkeykiri-plltxe-frapo-everyone-talks-about-the-weather/}

\subsection{H\`i(’)-} Abgeleitet von dem Adjektiv \N{h\`i’i}, \D{klein}, wird dieser
Pr\"afix in einigen wenigen W\"ortern eingesetzt, um Verkleinerungsformen zu bilden,
sollte aber nicht als produktiv angesehen werden (siehe auch \horenref{lingop:dimin}),
wie in \N{\ACC{h\`i}’ang}, \D{Insekt} ($<$ \N{h\`i’} $+$ \N{ioang},
\D{Tier}), \N{\ACC{h\`i}krr}, \D{Moment, Augenblick} ($<$ \N{h\`i} $+$
\N{krr}, \D{Zeit}).\index{hiì(')-@\textbf{h\`i(’)-}}

\subsection{-\`iva} Wenn das Substantiv \N{\`ilva}, also \D{Flocke, Tropfen, Schnitzel}
in zusammengesetzten W\"ortern verwendet wird, entf\"allt das \N{-l-}. \N{\ACC{txe}p\`iva},
\D{Asche, Schlacke} oder \N{\ACC{her}w\`iva}, \D{Schneeflocke}.
\NTeri{1/4/2011}{http://naviteri.org/2011/04/yafkeykiri-plltxe-frapo-everyone-talks-about-the-weather/}
\index{-iva@\textbf{-\`iva}}\index{ilva@\textbf{\`ilva}}

\subsection{-nga’} Dieser Suffix, der von dem Verb \N{nga’} abgeleitet wird, erzeugt
Adjektive aus Substantiven, die etwas beschreiben, das das Substantiv "`enth\"alt"',
wie in \N{\ACC{txum}nga’} \D{giftig}. Er ist weitaus weniger h\"aufig als \N{le-}.
Es ist m\"oglich, da\ss{} ein Substantiv sowohl mittels \N{le-} als
auch mittels \N{nga’} abgeleitet wird. \N{lepay} \D{w\"a\ss{}rig} gegen\"uber
\N{\ACC{pay}nga’} \D{dunstig, feucht}.
\index{-nga'@\textbf{-nga’}}
\NTeri{5/5/2011}{http://naviteri.org/2011/05/weather-part-2-and-a-bit-more-2/}

\subsection{-pin} Dieser unbetonte Suffix, das von dem Substantiv \N{’opin}, \D{Farbe},
abgeleitet wird, wird an Farbadjektive geh\"angt, um Farbsubstantive zu bilden.
\N{\ACC{rim}pin}, \D{Gelb (Subst.)} von \N{rim}, \D{gelb (Adj.)}.
Endet das Farbadjektiv mit einem \N{-n}, so wird es zu einem \N{-m} angepa\ss{}t.
\N{\ACC{e}ampin} von \N{\ACC{e}an}.\index{'opin@\textbf{’opin}}\index{-pin@\textbf{-pin}}

\subsection{Pxi-} Das Adjektiv \N{pxi}, \D{scharf}, wird Adverben und Adpositionen der
Zeit vorangestellt, um Unmittelbarkeit auszudr\"ucken. Der Pr\"afix bekommt dabei nicht
die Betonung.
\N{pxi\ACC{sre}}, \D{direkt vorher} und \N{pxi\ACC{set}}, \D{genau jetzt}.
\index{pxi-@\textbf{pxi-}}

\subsection{Sna-} Eine Kurzform des Substantivs \N{sna’o}, \D{Gruppe, Ansammlung, Klumpen, Standplatz}.
Dieser Pr\"afix kann ohne Einschr\"ankungen mit lebenden Wesen, bei denen es sich nicht um
Personen handelt, verwendet werden, um eine nat\"urliche Gruppierung anzuzeigen, so wie
\N{snatalioang}, \D{eine Sturmbeestherde} oder \N{snautra}, \D{eine Baumgruppe}.
Dieser Pr\"afix wird mit leblosen Dingen verwendet, um W\"orter zu erzeugen, aber dies ist
nicht produktiv: \N{snatx\"arem}, \D{Skelett}.
\NTeri{31/3/2012}{http://naviteri.org/2012/03/spring-vocabulary-part-2/}
\index{sna-@\textbf{sna-}}

\subsection{Tsuk-} Dieser unbetonte Pr\"afix, der von \N{tsun fko} abgeleitet wird, erzeugt
Adjektive, die eine Eignung oder Qualifikation ausdr\"ucken, aus transitiven Verben:
\N{tsuk\ACC{yom}}, \D{e\ss{}bar} (von \N{yom}, \D{essen}).
F\"ur die Negation wird einfach das \N{ke-} vorangestellt, wodurch ebenfalls keine
Verschiebung der Betonung stattfindet: \N{ketsuk\ACC{tswa’}},
\D{unverge\ss{}lich} (von \N{tswa’}, \D{vergessen}).
\index{tsuk-@\textbf{tsuk-}}\index{ketsuk-@\textbf{ketsuk-}}
\NTeri{22/3/2011}{http://naviteri.org/2011/03/\%E2\%80\%9Creceptive-ability\%E2\%80\%9D-and-hesitation/}

\subsubsection{} Zus\"atzlich k\"onnen intransitive Verben mit
\N{tsuk-} kombiniert werden, jedoch mit einer schw\"achere Beziehung zwischen dem
Substantiv und dem erzeugten Adjektiv.
\Npawl{F\`itseng lu tsuk\-tsurokx}, \D{Man kann hier rasten}
\Npawl{Lu na’r\`ing tsukhahaw} \D{Man kann im Wald schlafen}.

\subsection{-tswo} Dieser Suffix kann ohne Einschr\"ankungen mit jedem Verb verwendet
werden und erzeugt ein Substantiv, das die F\"ahigkeit, die Aktion des Verbs auszuf\"uhren,
ausdr\"uckt: \N{wemtswo}, \D{Kampfesfertigkeit} oder \N{roltswo} \D{Gesangsfertigkeit}.
Er ist verwandt mit dem Wort \N{tsu’o}, \D{F\"ahigkeit, Fertigkeit}.
\NTeri{31/3/2012}{http://naviteri.org/2012/03/spring-vocabulary-part-2/}
\index{-tswo@\textbf{-tswo}}

\subsubsection{} Der Suffix \N{-tswo} wird an die Substantiv- oder Adjektivkomponente
von mit \N{si} konstruierten Verben geh\"angt, wie in \N{srung\-tswo}, \D{F\"ahigkeit zu
helfen} oder \N{tstutswo}, \D{F\"ahigkeit zu verschlie\ss{}en}.

\subsection{-vi} Abgeleitet von \N{’evi}, \D{Kind}, der Kurzform von \N{’eveng}, \D{Kind}.
Der unbetonte Suffix \N{-vi} wird verwendet, um eher im \"ubertragenen Sinne eine
Abspaltung von etwas Gr\"o\ss{}erem oder einen Teil eines gr\"o\ss{}eren Ganzen zu
definieren.
\N{\ACC{txep}vi} \D{Funke} ($<$ \N{txep} \D{Feuer}), \N{\ACC{l\`i’}fyavi}
\D{Ausdruck, Teil einer Sprache} ($<$ \N{l\`i’fya} \D{Sprache}).
Er kann kleinere Ver\"anderungen des Wortes, an das er geh\"angt wird, hervorrufen.
\N{s\"a\ACC{num}vi} \D{Lektion, Unterrichtsstunde} von \N{s\"a\ACC{nu}me} \D{Lehre,
Unterweisung}.\index{-vi@\textbf{-vi}}
\LNWiki{14/3/2010}{http://wiki.learnnavi.org/index.php/Canon/2010/March-June\%23A_Collection}

\subsection{"`K\"a-"' und "`Za-"'} Die beiden Verben der Bewegung \N{k\"a}, \D{gehen}
und \N{za’u}, \D{kommen} (zu \N{za-} verk\"urzt) werden in einigen zusammengesetzten
Verben verwendet, um die Orientierung einer Bewegung anzuzeigen.
\N{k\"a\ACC{mak}to}, \D{ausreiten}. Beachten Sie bitte die Unterscheidung zwischen
\N{k\"a\ACC{’\"a}r\`ip}, \D{schieben} und \N{za\ACC{’\"a}r\`ip}, \D{ziehen}, abgeleitet von
\N{\ACC{’\"a}r\`ip} \D{(etwas) bewegen}.
\index{kaä-@\textbf{k\"a-}}\index{za-@\textbf{za-}}


\section{Zeit}
\noindent Zeitadverben werden auf regelm\"a\ss{}ige Art und Weise abgeleitet.

\subsection{Gleichzeitigkeit} Die Vorsilbe \N{f\`i-} erzeugt ein Adverb, das sich
auf die aktuelle Zeiteinheit bezieht.
\N{f\`itrr}, \D{heute} (= "`dieser Tag"') oder \N{f\`irewon}, \D{dieser Morgen}.
\index{fiì-@\textbf{f\`i-}!in Zeitadverben}

\subsection{Vorzeitigkeit} Der betonte Suffix \N{-am} erzeugt ein Adverb, das sich
auf eine vergangene Zeiteinheit bezieht.
\N{trr\ACC{am}}, \D{gestern} oder \N{pxiswaw\ACC{am}} \D{einen Moment zuvor}.
\index{-am@\textbf{-am}}

\subsection{Nachzeitigkeit} Der betonte Suffix \N{-ay} erzeugt ein Adverb, das sich
auf eine Zeiteinheit in der Zukunft bezieht.
\N{trr\ACC{ay}}, \D{morgen} oder \N{ha’ngir\ACC{ay}}, \D{morgen nachmittag}.
\index{-ay@\textbf{-ay}}