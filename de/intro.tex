\nchapter{Einf\"uhrung}

Bisher existiert noch keine offizielle Grammatik der Sprache der Na’vi,
die von Paul Frommer mit dem Segen von Lightstorm Entertainment und 20th
Century Fox geschrieben wurde. Als das Original verfa\ss{}t wurde
\footnote{Juli 2010}, erschien es unwahrscheinlich, da\ss{} wir auf
absehbare Zeit eine bek\"amen. Daher hatte William S. Annis eine von ihm
verfa\ss{}te Zusammenfassung der Grammatik ausf\"uhrlicher gestaltet.

Wie jene Zusammenfassung wird dieses Werk Ihnen das Na’vi nicht beibringen,
sondern es soll ein pr\"azises und korrektes Nachschlagewerk darstellen,
das den aktuellen Wissensstand \"uber diese Sprache widerspiegelt. Es 
basiert auf s\"amtlichen Analysen, die nach der Ver\"offentlichung des
Films stattgefunden hatten, sowie diversen Mitteilungen Frommers, die
bestimmte Aspekte der Sprache beleuchten.

Dies basiert weitestgehend auf den Seiten des LearnNavi.org-Wikis, ohne
deren Informationen dieses Werk nicht m\"oglich gewesen w\"are. Frommers
eigenes Blog stellt ebenfalls reichhaltiges Material
bereit.\footnote{Erstellt Ende Juni 2010, \url{http://naviteri.org}}


\section{Geschichte der Entschl\"usselung}
F\"ur Neulinge ist es wichtig zu verstehen, weshalb wir das \"uber die
Sprache der Na’vi wissen, was wir wissen.

Die ersten Hinweise zu dieser Sprache tauchten in Interviews mit Frommer
bereits im Dezember 2009 und bis zur Ver\"offentlichung des Films auf.
Na’vi verf\"ugte \"uber Ejektive. Es verf\"ugte \"uber eine dreigeteilte
Fallunterscheidung. Einige wenige S\"atze waren bekannt.

Die gro\ss{}e Chance ergab sich, als jemand von denen, die die IMDB
verlassen hatten, die Liste der Na’vi-W\"orter\footnote{\url{
http://kcbluesman.websitetoolbox.com/post?id=4013403}, Login notwendig}
gepostet hatten. Sie wurde dem \textit{Activist Survival Guide}\footnote{
Wihelm, Maria; Mathison, Dirk (2009). \textit{James Cameron’s Avatar: A
Confidential Report on the Biological and Social History of Pandora (An
Activist Survival Guide),} It Books (Harper Collins).} entnommen.
Jene Liste wurde in einem Blogbeitrag am 11. Dezember erneut
ver\"offentlicht.\footnote{\url{http://www.suburbandestiny.com/?p=611}}
Alle aktuellen W\"orterb\"ucher basieren auf diesem ersten Beitrag. Dies
ergab ein gen\"ugend umfangreiches Vokabular, um anfangen zu k\"onnen,
die S\"atze, die in den Interviews mit Frommer auftauchten, zu analysieren.

Am 15. Dezember erschien in einem Interview mit dem UGO-Filmblog\footnote{
\url{http://www.ugo.com/movies/paul-frommer-interview}} erstmalig die
grundlegende Gru\ss{}formel der Na’vi, \N{oel ngati kameie}, \D{Ich Sehe dich}.
Das war dazu die erste Sichtung der Fallendungen f\"ur den Agens und den
Patiens. Dank des W\"orterbuchs konnten wir \N{-l} als Endung f\"ur den
Agens und \N{-ti} als Endung f\"ur den Patiens ermitteln.

Unser n\"achster gro\ss{}er Durchbruch stellte sich ein paar Tage sp\"ater,
am 19. Dezember, mit einem Gastbeitrag in Form einer Sprachkladde.\footnote{
\url{http://languagelog.ldc.upenn.edu/nll/?p=1977}}
Dies ist immer noch grundlegende Lekt\"ure f\"ur jeden Studenten des Na’vi.
Darin erfahren wir eine Menge \"uber das phonetische System des Na’vi, und
es verriet uns zudem genug \"uber die Grammatik des Na’vi, um uns bei
zuk\"unftigen Analysen der Beispiele, die in sp\"ateren Interviews
auftauchten, zu helfen.

Selbst jetzt stammt das meiste Wissen, das wir haben, nicht von Frommer
selbst, beispielsweise in Form von "`Dies ist die Endung des Genitiv"',
sondern weil er in einem Interview sagte, da\ss{} es einen Genitiv g\"abe
und da\ss{} man diese Information nutzte, um die Sprachbeispiele des Na’vi
zu analysieren. Einige der anf\"anglichen Analysen waren jedoch
unvollst\"andig, was zu einiger Verwirrung, insbesondere was die Fallendungen
betrifft, gef\"uhrt hatte. Die allerersten Beispiele enthielten alle die
Endung \N{-y\"a}. Erst sp\"ater ergaben sich Hinweise auf die Existenz der
Endung \N{-\"a}. Noch immer gibt es Nachschlagewerke, die lediglich die
Endung \N{-y\"a} f\"ur den Genitiv auff\"uhren.

In den darauffolgenden Monaten lieferte Frommer selbst umfangreichere
Beispiele des Na’vi, von denen jedes m\"oglichst detailgenau analysiert
wurde, um m\"oglichst viel Informationen zur Grammatik herauszubekommen.
Ferner hat Frommer einige direkte Fragen zu der Sprache beantwortet. Oftmals
best\"atigt er damit das, was wir nach den Analysen bereits vermuteten,
manchmal lassen sich so Irrt\"umer korrigieren, und manchmal erhalten wir so
neue Informationen.

Ich habe m\"oglichst versucht sicherzustellen, da\ss{} alle Teile der
Grammatik von Frommer selbst best\"atigt wurden, oder wenn das nicht m\"oglich
war, habe ich gen\"ugend Beispiele aus Frommers eigenem Na’vi angef\"uhrt,
um den grammatikalischen Aspekt zu verdeutlichen. Dennoch ist dieses Werk als
vorl\"aufig zu betrachten. Es ist Frommers Vorrecht, diese Sprache nach seinen
eigenen Vorstellungen davon, was daf\"ur notwendig ist, anzupassen
und zu aktualisieren, etwaige Mi\ss{}verst\"andnisse zu korrigieren, die
bisher seiner Aufmerksamkeit entgangen waren und grammatikalische L\"ucken zu
schlie\ss{}en, sobald er sich ihrer annimmt. Wir m\"ussen ebenfalls davon
ausgehen, da\ss{} zuk\"unftige \textit{Avatar}-Filme die Sprache auf
unerwartete Art und Weise ver\"andert, und zwar nicht nur, um Camerons
Anforderungen an seine Filme zu befriedigen, sondern aufgrund der
unvermeidlichen Tatsache, da\ss{} eine k\"unstliche Sprache \"Anderungen
unterworfen ist, wenn Schauspieler sie auf der B\"uhne sprechen.


\section{Notationen und Konventionen}

Text auf Na’vi wird grunds\"atzlich \textbf{fett} und die deutschen \"Ubersetzungen
\textit{kursiv} geschrieben: \N{f\`ifya} -- \D{daher}.

Wenn ein Beispiel f\"ur das Na’vi direkt und unver\"anderungen den Interviews,
E-Mails oder dem Blog Frommers entstammt, wird ein $\mathcal{F}$ im Rand stehen,
wie bei \Npawl{k\`iyevame}. Das \textit{Jagdlied} und das \textit{Weberlied} aus
dem \textit{Activist Survival Guide} sind ebenfalls so markiert. Beispiele aus
dem Film verwenden $\mathcal{A}$.

Dieses Werk nutzt die Digraphen \N{ts} und \N{ng} anstelle der wissenschaftlichen
Rechtschreibung, die von Frommer entwickelt worden ist (\horenref{l-and-s:cg}).
Die meisten Leute sind mit dem Digraph-System vertrauter.

Die Betonung wurde in Frommers Dokumentation dadurch angezeigt, da\ss{} die betonte
Silbe unterstrichen war. Dieses Werk beh\"alt diese Praktik bei, wie in
\N{\ACC{tu}te} -- \D{Person} gegen\"uber \N{tu\ACC{te}} -- \D{Frau}.
Um Kollisionen mit Frommers Konventionen zur Betonung zu vermeiden wird eine
wellige \uwave{Unterstreichung} verwendet, um die Aufmerksamkeit auf bestimmte
Teile von W\"ortern oder S\"atzen zu lenken.

Zudem finden die \"ublichen Konventionen anderer linguistischer Werke Anwendung.
Hypothetische Beispiele oder welche, die irgendeinen Fehler enthalten, werden
mit einem vorangestellten Stern markiert, wie in *\N{m’resh’tuyu}. Pr\"afixe werden
mit einem Bindestrich am Ende markiert, wie bei \N{f\`i-}, und wenn ein Pr\"afix
Lenisierung ausl\"ost (\horenref{l-and-s:lenition}) wird ein Pluszeichen
nachgestellt, so wie bei \N{ay+}. Suffixe werden mit einem f\"uhrenden Bindestrich
markiert, wie z. B. bei \N{-it} und Infixe mit spitzen Klammern, wie bei
\N{\INF{ol}}. Angaben im internationalen phonetischen Alphabet stehen in eckigen
Klammern, wie bei {\gplus [fɪ.ˈfja]}.

Wird eines der vier Lieder, die Frommer f\"ur den Film \"ubersetzt hat, zitiert,
so werden die einzelnen Verse durch einen einzelnen Schr\"agstrich getrennt, wie
bei \N{Rerol tengkrr ker\"a / \`Il\"a fya’o avol}.

Beginnend mit September 2011 werden f\"ur neues Material Verweise zu
grammatikalischen Einzelheiten mit angegeben. Sie stehen am Ende eines Abschnitts
und sehen wie folgt aus:
\NTeri{11/7/2010}{http://naviteri.org/2010/07/diminutives-conversational-expressions/}.
Bitte beachten Sie, da\ss{} die Angaben von Daten der deutschen Konvention
entsprechen, also TT.MM.JJJJ.

"`NT"' steht f\"ur Frommers Blog, inklusive seiner Antworten in Kommentaren.
"`Wiki"' steht f\"ur das Wiki von LN.org und "`Ultxa"' f\"ur das Treffen im
Oktober 2010. F\"ur eine vollst\"andige Erfassung aller Zitate werden
wahrscheinlich einige Monate vergehen.

Text in \QUAESTIO{Magenta} zeigt Angelegenheiten auf, die nach meinem Erachten
wichtige Fragen zur Sprache darstellen, f\"ur die bisher jedoch keine Informationen
vorliegen. Einige bed\"urfen lediglich einer Best\"atigung durch Frommer, w\"ahrend
andere wesentlich genauere \"Uberlegungen und Arbeit seinerseits erfordern. Dieses
Werk strebt an, irgendwann frei von magentafarbenem Text zu sein.

\bigskip

\bigskip
Dank geht an die Mitglieder von LearnNavi.org ’Eylan Alfalulukan\"a,
Taronyu und Ftiafpi daf\"ur, da\ss{} sie die Entw\"urfe dieses Werks gegengelesen
und Vorschl\"age angebracht haben. Ich habe jedoch nicht immer ihren Rat befolgt,
daher liegt die Verantwortung f\"ur etwaige Schw\"achen bei mir.

Dank geht ferner an alle, die dieses Werk kommentiert und Korrekturen angebracht
haben, seit dieses Werk erstmalig ver\"offentlicht wurde.