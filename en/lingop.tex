\nchapter{Word Building}

\section{Derivational Affixes}\index{affix!derivational}
\noindent Na'vi has a number of affixes used to create new vocabulary.
Several simply change the word class, such turning a noun into
adjective.  However, these affixes should not be considered freely
productive, and the meanings of the derived forms are not entirely
predictable.\label{lingop:affixes} Only with the help of a dictionary
can you be certain of a derived word's meaning (but see
\horenref{syn:nifyao} for adverbs).  Unless otherwise stated, the
affixes below are not freely productive.

\medskip
\noindent While there are strong patterns in how stress accent is
altered by some derivational processes, there are no exceptionless
rules for this.  Again, only with the dictionary can you be certain of
the accenting of a derived word.


\subsection{Prefixes} These derivational prefixes rarely cause the
accent to move from its original location, \N{ngay} $>$ \N{tì\ACC{ngay}}.

\subsubsection{} \N{Le-} creates adjectives from nouns, as in
\N{le\ACC{hrr}ap} \E{dangerous} from \N{\ACC{hrr}ap} \E{danger}.
\index{le-@\textbf{le-}}

\subsubsection{} \N{Nì-} creates adverbs from nouns, pronouns,
adjectives and verbs, as in \N{nìNa'vi} \E{Na'vily, in Navi} from
\N{Na'vi}, \N{nìayfo} \E{like them}, \N{nì\ACC{ftu}e} \E{easily} from
\N{\ACC{ftu}e} \E{easy}, and \N{nì\ACC{tam}} \E{enough} from \N{tam}
\E{to suffice}.  \index{niì-@\textbf{nì-}}

\subsubsection{} \N{Sä-} creates instrumental nouns from verbs and
adjectives, as in \N{sä\ACC{nu}me} \E{instruction, teaching} from
\N{\ACC{nu}me}, and \N{sä\ACC{spxin}} \E{disease} from \N{spxin}
\E{sick}. \index{saä-@\textbf{sä-}}

\subsubsection{} \N{Sä-} also creates nouns to indicate a particular,
concrete instance of a general action. A \N{sätsyìl} \E{a climb} is a
particular instance of the action of climbing, \N{tsyìl}.  Roots may
have derivations in both \N{tì-} and \N{sä-}, as in \N{'ipu}
\E{humorous}.  \N{Tì'ipu} is the abstract concept of being humorous,
that is, humor in general.  \N{Sä'ipu} is a particular instance of being
humorous --- for example, a joke.\\
\NTeri{29/2/2012}{http://naviteri.org/2012/02/trr-asawnung-lefpom-happy-leap-day/}

\subsubsection{} \N{Tì-} creates nouns from adjectives, verbs and
occasionally other nouns, as in \N{tì\ACC{ngay}} \E{truth} from
\N{ngay} \E{true}, \N{tìfti\ACC{a}} \E{study (n.)} from
\N{fti\ACC{a}} \E{to study}, \N{tì\ACC{'awm}} \E{camping} from \N{'awm}
\E{camp (n.)}. \index{tiì-@\textbf{tì-}}

\subsubsection{} With these prefixes stem syllables may lose a vowel
if the onset consonant is also a legal coda, \N{nìm\ACC{wey}pey}
\E{patiently} $<$ \N{ma\ACC{wey}·pey} \E{to be patient}.


\subsection{Negative Prefix} Some words, mostly but not exclusively
adjectives, are created using the word \N{ke} \E{not} as a prefix.

\subsubsection{} When \N{ke-} comes before the adjective prefix
\N{le-} the adjective prefix is reduced to just \N{-l-}, as in
\N{kel\ACC{tsun}} \E{impossible} compared to \N{le\ACC{tsun}slu}
\E{possible}, and \N{kel\ACC{fpom}tokx} \E{unhealthy} from
\N{lefpom\ACC{tokx}} \E{healthy}.

\subsubsection{} When \N{le-} comes before \N{ke-} the negative prefix
reduces to just \N{-k-}, as in \N{lek\ACC{ye}'ung} \E{insane} from
\N{ke\ACC{ye}'ung} \E{insanity}.

\subsubsection{} The \N{ke-} prefix may be used with root adjectives
and participles, in which case the accent usually shifts to \N{ke-},
as in \N{\ACC{ke}teng} \E{different} from \N{teng} \E{same, equal} and
\N{\ACC{ke}rusey} \E{dead} from \N{ru\ACC{sey}} \E{living}.  However,
note \N{key\ACC{awr}} \E{incorrect} from \N{ey\ACC{awr}} \E{correct}.
\label{lingop:prefix:ke}

\subsubsection{} The \N{ke-} prefix may also create and combine with
nouns, as in \N{ke\ACC{ye}'ung} \E{insanity}, and \N{\ACC{ke}tuwong}
\E{alien}.  There are too few examples to determine accent behavior.


\subsection{Adverbial ``a-''} Two stative verbs, \N{lìm} \E{be far}
and \N{sim} \E{be near} have adverbial forms \N{a\ACC{lìm}} \E{far
away} and \N{a\ACC{sim}} \E{nearby, at close range}.  These are
thought of as fossilized abbreviations of forms like \N{nìfya'o a
lìm} (\horenref{syn:nifyao}).  They are fixed lexical items, and do
not have forms such as \N{*lìma} and \N{*sima}.
\index{-a-@\textbf{-a-}!with adverbs}\index{aliìm@\textbf{alìm}}\index{asim@\textbf{asim}}
\LNWiki{17/5/2010}{http://wiki.learnnavi.org/index.php/Canon/2010/March-June\%23Near.2C_Distant_and_Irregular_Adverbs}


\subsection{Prefix with Infix} There is a single derivation using the
combination of a prefix and an infix.

\subsubsection{} \N{Tì- ‹us›} creates a gerund.  It is fully
productive for verb roots and compounds (\N{si}-con\-struc\-tion verbs,
\horenref{lingop:si-const}, cannot be made into a gerund).  This is
most useful when a simple \N{tì-} derivation already has an
established meaning, as in \N{rey} \E{live}, \N{tì\ACC{rey}} \E{life},
but \N{tìru\ACC{sey}} \E{living}.  In compounds, \N{tì-} comes at the
beginning of the word and \N{‹us›} goes into the verbal element of the
compound, \N{yomtìng} becomes \N{tìyomtusìng}. See also
\horenref{syn:gerund}.
\index{gerund!formation}\label{lingop:gerund}
\index{si construction@\textbf{si} construction!no gerund}
\LNForum{31/1/2013}{https://forum.learnnavi.org/language-updates/confirmations-(comparisons-gerund)/msg572997/}


\subsection{Agent Suffixes} These suffixes also do not cause an accent
shift.

\subsubsection{} \N{-tu} creates agent nouns from parts of speech
other than verbs, as in \N{\ACC{pam}tseotu} \E{musician} from
\N{\ACC{pam}tseo} \E{music}, \N{tsul\ACC{fä}tu} \E{master of a craft
or skill, expert} from \N{tsul\ACC{fä}} \E{mastery}.\index{-tu@\textbf{-tu}}

\subsubsection{} \N{-yu} creates agent nouns from verbs, to indicate a
person who regularly performs some activity or role, as in
\N{taronyu} \E{hunter} from \N{taron} \E{hunt}.  This suffix is freely
productive, with regular verbs as well as \N{si}-verbs, such
as \N{stiwisiyu} \E{mischief-maker,} from \N{stiwi si} \E{do mischief}.
\index{-yu@\textbf{-yu}}
\NTeri{11/7/2010}{http://naviteri.org/2010/07/diminutives-conversational-expressions/}
\LNForum{30/10/2020}{https://forum.learnnavi.org/language-updates/yu-is-officially-productive-on-si-verbs/}


\subsection{Diminutive Suffix} The unstressed suffix \N{-tsyìp} may be
used freely to form diminutives, on both nouns and pronouns.  Personal
names may lose syllables when taking this suffix, such as \N{Kamtsyìp} or
\N{Kamuntsyìp} for \E{little Kamun}.  The diminutive has three uses.
\label{lingop:dimin}\index{diminutive}\index{tsyiìp@\textbf{-tsyìp}}
\NTeri{11/7/2010}{http://naviteri.org/2010/07/diminutives-conversational-expressions/}

\subsubsection{} First, the diminutive form may be a primary lexical
derivation.  Such words will end up in the dictionary, such as
\N{puktsyìp} \E{booklet, pamphlet} from \N{puk} \E{book}.  The
diminutive force is weak enough that one may use the adjective
\N{tsawl} \E{large} with a diminutive without contradiction, as in
\N{tsawla utraltsyìp} \E{a large bush}.

\subsubsection{} Second, the diminutive may express affection or
endearment, \Npawl{za'u fì\-tseng, ma 'itetsyìp} \E{come here, little
daughter}. This use should not be taken to imply an age.  The daughter
in the previous sentence could be an adult.

\subsubsection{} Third, the diminutive may express disparagement or
insult, \Npawl{fìtaron\-yu\-tsyìp ke tsun ke'ut stivä'nì} \E{this
(worthless) little hunter can’t catch anything}.  The disparaging
tone may be directed at oneself, \Npawl{nga nìawnomum to
\uwave{oetsyìp} lu txur nìtxan} \E{as everyone knows, you're a lot
stronger than \uwave{little old me}}.  Only context will distinguish
the disparaging from affectionate use of the diminutive.


\subsection{-Nay Suffix} This creates a new noun which indicates
something lower on some hierarchy, size, rank, accomplishment, etc.
The suffix receives the accent, \N{karyu\ACC{nay}} \E{apprentice
teacher} from \N{karyu} \E{teacher}.  If the noun already ends in
\N{-n} the suffix loses the \N{-n-}, \N{'eyla\ACC{nay}}
\E{acquaintance} from \N{'eylan} \E{friend}, \N{ikra\ACC{nay}}
\E{forest banshee} from \N{ikran} \E{banshee}.  It isn't productive.
\index{-nay@\textbf{-nay}}
\NTeri{2/28/2013}{http://naviteri.org/2013/02/vospxi-ayol-posti-apup-short-post-for-a-short-month/}


\subsection{Gender Suffixes} The gender suffixes are unusual in that
they are used not only with nouns but also the third person pronoun
(\horenref{morph:pron:gender}).  \label{lingop:suffix:gender}

\subsubsection{} The suffix \N{-an} indicates males, as in
\N{po\ACC{an}} \E{he} and \N{\ACC{'i}tan} \E{son}.

\subsubsection{} The suffix \N{-e} indicates females, as in
\N{po\ACC{e}} \E{she} and \N{\ACC{'i}te} \E{daughter}.

\subsubsection{} The effect of these suffixes on the accent is
unpredictable, \N{tu\ACC{tan}} \E{male (person)} from \N{\ACC{tu}te}
\E{person,} but \N{mun\ACC{txa}tan} \E{husband} from
\N{mun\ACC{txa}tu} \E{spouse, mate}.


\section{Reduplication}
\noindent Reduplication is a nonproductive derivational process.
Nonetheless, a few common words do use it. \index{reduplication}

\subsection{Iteration} With words of time, reduplication indicates
repetition or habitual occurrence, \N{letrrtrr} \E{ordinary,} that is,
occurring daily; and \N{krro krro} \E{sometimes}.

\subsection{Shift in Degree} With the verbs \N{'ul} \E{increase} and
\N{nän} \E{decrease}, reduplicated adverbs mark change to an extreme
degree, \N{nì'ul'ul} \E{increasingly, more and more,}
\N{nìnänän}\footnote{The reduplication is partial, since consonants
cannot be doubled.} \E{less and less}.\\
\NTeri{29/2/2012}{http://naviteri.org/2012/02/trr-asawnung-lefpom-happy-leap-day/}

\section{Compounds}

\subsection{Headedness} The dominant element of a Na'vi compound may
come first or last in the compound.\footnote{Many human languages are
more strict.  English compounds, for example, generally have the
dominant element, or ``head,'' last, as in \textit{blueberry},
\textit{night-light}, \textit{blackboard}.  On the other hand,
Vietnamese uses head-initial order for native compounds and
head-final order for compounds using the substantial Chinese
vocabulary it has borrowed.}  There is, however, a strong tendency
for head-final compounds.  Verb compounds are the most likely to be
head-initial.

\subsubsection{} Compounds are in the same word class as their head,
so \N{txam\ACC{pay}} \E{sea} is noun, because \N{pay} \E{water} is a
noun.

\subsubsection{} Like root words, compounds may change word class with
the addition of the derivational affixes listed above,
\N{lefpom\ACC{tokx}} \E{healthy} from \N{fpom\ACC{tokx}} \E{health}.


\subsection{Apocope} Words may lose parts when used in a compound, as
in \N{\ACC{ven}zek} \E{toe} $<$ \N{\ACC{ve}nu} \E{foot} $+$
\N{\ACC{zek}wä} \E{finger}, and \N{sìl\ACC{pey}} \E{hope} $<$
\N{sìltsan} \E{good} $+$ \N{pey} \E{wait (for)}.


\subsection{``Si'' Construction} The usual way to convert a noun or
adjective to a verb is to pair the uninflected noun with the prop verb
\N{si}, which only ever occurs in these constructions.  The order is
fixed \N{N si}, with \N{si} getting all verb affixes.\label{lingop:si-const}
\index{si construction@\textbf{si} construction}

\subsubsection{} In the verb \N{irayo si} \E{to thank} the order is
less fixed.
\LNWiki{12/5/2010}{http://wiki.learnnavi.org/index.php/Canon/2010/March-June\%23Word_Order_Issues}

\subsubsection{} The normal \N{N si} word order is also broken for
negation, \N{oe pamrel ke si} \E{I don't write} (\horenref{syn:neg:si-const}),
\N{txopu rä'ä si} \E{don't be afraid} (\horenref{syntax:prohibitions}).


\section{Common and Noteworthy Compound Elements}

\subsection{-fkeyk} Derived from the noun \N{tìfkeytok} \E{state,
condition, situation}, this unaccented suffix produces some words with
specialized, idiomatic meanings, such as \N{\ACC{ya}fkeyk}
\E{weather}.  It is none\-the\-less widely productive, \Npawl{kilvanfkeyk
lu fyape fìtrr?} \E{how's the condition of the river today?}
\index{-fkeyk@\textbf{-fkeyk}}
\NTeri{1/4/2011}{http://naviteri.org/2011/04/yafkeykiri-plltxe-frapo-everyone-talks-about-the-weather/}

\subsection{Hì(')-} From the adjective \N{hì'i} \E{small}, the accented
prefix \N{hì-} or \N{hì'-} is used in a few words to form diminutives,
but should not be considered productive (see \horenref{lingop:dimin}),
as in \N{\ACC{hì}'ang} \E{insect} ($<$ \N{hì'} + \N{ioang}
\E{animal}), \N{\ACC{hì}krr} \E{moment, a short time} ($<$ \N{hì} +
\N{krr} \E{time}).  \index{hiì(')-@\textbf{hì(')-}}

\subsection{-ìva} When the noun \N{ìlva} \E{flake, drop, chip} is used
in compounds, the \N{l} drops, \N{\ACC{txe}pìva} \E{ash, cinder,}
\N{\ACC{her}wìva} \E{snowflake}.
\NTeri{1/4/2011}{http://naviteri.org/2011/04/yafkeykiri-plltxe-frapo-everyone-talks-about-the-weather/}
\index{-iva@\textbf{-ìva}}\index{ilva@\textbf{ìlva}}

\subsection{-nga'} This suffix, derived from the verb \N{nga'}
\E{contain}, creates adjectives from nouns and describes something
``containing'' the noun, as in \N{\ACC{txum}nga'} \E{poisonous}.  It
is much less common than \N{le-}.  It is possible for a noun to have
both \N{le-} and \N{-nga'} derivations, \N{lepay} \E{watery} vs.\
\N{\ACC{pay}nga'} \E{damp, humid}.
\index{-nga'@\textbf{-nga'}}
\NTeri{5/5/2011}{http://naviteri.org/2011/05/weather-part-2-and-a-bit-more-2/}

\subsection{-pin} Derived from the noun \N{'opin} \E{color}, this
unaccented suffix is attached to color ad\-jec\-tives to form color nouns,
\N{\ACC{rim}pin} \E{the color yellow} from \N{rim} \E{yellow}.  A
final \N{-n} in the color adjective will become \N{-m} by
assimilation, \N{\ACC{e}ampin} from \N{\ACC{e}an}.
\index{'opin@\textbf{'opin}}\index{-pin@\textbf{-pin}}

\subsection{Pxi-} The adjective \N{pxi} \E{sharp} is prefixed to
adverbs and adpositions of time to indicate immediacy.  The prefix
doesn't take the accent, \N{pxi\ACC{sre}} \E{immediately before},
\N{pxi\ACC{set}} \E{immediately, right now}.
\index{pxi-@\textbf{pxi-}}

\subsection{Sna-} A shortened form of the noun \N{sna'o} \E{group,
set, clump, stand}, this prefix can be freely used with living
things other than people to indicate a natural grouping, such as
\N{snatalioang} \E{a herd of sturmbeest}, \N{snautral} \E{a stand of
trees}.  The prefix is used with non-living things to produce words,
but this is not productive, \N{snatxärem} \E{skeleton}.
\NTeri{31/3/2012}{http://naviteri.org/2012/03/spring-vocabulary-part-2/}
\index{sna-@\textbf{sna-}}

\subsection{Tsuk-} Derived from \N{tsun fko}, this unaccented prefix
creates ability adjectives from transitive verbs, \N{tsuk\ACC{yom}}
\E{edible} (from \N{yom} \E{eat}).  The negative simply takes the
prefix \N{ke-}, which also causes no accent change here,
\N{ketsuk\ACC{tswa'}} \E{unforgettable} (from \N{tswa'} \E{forget}).
\index{tsuk-@\textbf{tsuk-}}\index{ketsuk-@\textbf{ketsuk-}}
\NTeri{22/3/2011}{http://naviteri.org/2011/03/\%E2\%80\%9Creceptive-ability\%E2\%80\%9D-and-hesitation/}

\subsubsection{} In addition, intransitive verbs may
be combined with \N{tsuk-}, with a looser relationship between the
noun and resulting adjective, \Npawl{fìtseng lu tsuk\-tsurokx}
\E{one can rest here, this place is “restable,''} \Npawl{lu na’rìng
tsukhahaw} \E{one can sleep in the forest}.

\subsection{-tswo} This suffix may be freely used on any verb, and
creates a noun meaning the ability to perform the action of the verb,
\N{wemtswo} \E{ability to fight,} \N{roltswo} \E{ability to sing}.
This suffix is related to the word \N{tsu'o} \E{ability}.
\NTeri{31/3/2012}{http://naviteri.org/2012/03/spring-vocabulary-part-2/}
\index{-tswo@\textbf{-tswo}}

\subsubsection{} The suffix \N{-tswo} is attached to the noun or
adjective element of \N{si}-verbs, as in \N{srung\-tswo} \E{ability to
help} and \N{tstutswo} \E{ability to close}.

\subsection{-vi} From the noun \N{'evi}, itself a shortened form of
\N{'eveng} \E{child}, the unaccented suffix \N{-vi} is used rather
loosely for the spawn of something bigger or a part of a larger whole,
\N{\ACC{txep}vi} \E{spark} ($<$ \N{txep} \E{fire}), \N{\ACC{lì'}fyavi}
\E{expression, bit of language} ($<$ \N{lì'fya} \E{language}).  It may
cause minor changes to the word it is attached to, \N{sä\ACC{num}vi}
\E{lesson} from \N{sä\ACC{nu}me} \E{instruction, teaching}.
\index{-vi@\textbf{-vi}}
\LNWiki{14/3/2010}{http://wiki.learnnavi.org/index.php/Canon/2010/March-June\%23A_Collection}

\subsection{``Kä-'' and ``Za-''} The two verbs of motion \N{kä} \E{go}
and \N{za'u} \E{come} (reduced to just \N{za-}) are used in some
compound verbs to indicate direction of motion, \N{kä\ACC{mak}to}
\E{ride out}.  Note the distinction between \N{kä\ACC{'ä}rìp} \E{push}
and \N{za\ACC{'ä}rìp} \E{pull} from \N{\ACC{'ä}rìp} \E{move (something)}.
\index{kaä-@\textbf{kä-}}\index{za-@\textbf{za-}}


\section{Time}
\noindent Adverbs of time are derived from nouns in a predictable
pattern.

\subsection{The Current Time} The prenoun \N{fì-}
(\horenref{morph:prenoun:fi}) creates an adverb for the current unit
of time, \N{fìtrr} \E{today} (``this day''), \N{fìrewon} \E{this
morning}. \index{fiì-@\textbf{fì-}!in adverbs of time}

\subsection{The Previous Time} The accented suffix \N{-am} creates an
adverb for the previous unit of time, \N{trr\ACC{am}} \E{yesterday,}
\N{pxiswaw\ACC{am}} \E{just a moment ago}.
\index{-am@\textbf{-am}}

\subsection{The Next Time} The accented suffix \N{-ay} creates an
adverb for the next unit of time, \N{trr\ACC{ay}} \E{tomorrow}, 
\N{ha'ngir\ACC{ay}} \E{tomorrow afternoon}.
\index{-ay@\textbf{-ay}}
