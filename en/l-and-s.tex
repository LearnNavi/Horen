\nchapter{Letters and Sounds}


\section{Sound System}
\noindent The Na'vi language has 20 consonant sounds, 7 vowel sounds
and two vocalic resonants Frommer calls ``pseudovowels.''
\LanguageLog

\subsection{Consonants}

\begin{center}
\begin{tabular}{llllll}
 & Labial & Alveolar & Palatal & Velar & Glottal \\
Ejectives &	\N{px} [p'] & \N{tx} [t'] & & \N{kx} [k'] \\
Voiceless Stops & \N{p} [p] & \N{t} [t] & & \N{k} [k] & \N{’} [ʔ] \\
Affricate &             & \N{ts} {\gplus [t͡s]} \\
Voiceless fricatives & \N{f} [f] & \N{s} [s] & & & \N{h} [h] \\
Voiced fricatives & \N{v} [v] & \N{z} [z] \\
Nasals &         \N{m} [m] & \N{n} [n] & & \N{ng} [ŋ] \\
Liquids &         &  \N{r} [{\gplus ɾ}], \N{l} [l] \\
Glides &       \N{w} [w] & &  \N{y} [j] \\
\end{tabular}
\end{center}

\subsubsection{} The voiceless stops are unaspirated at the beginning
and middle of a word and un\-re\-lea\-sed at the end.  However, within a
phrase a final stop coming before a vowel will in natural speech be
released as the words flow together, \N{oel se\uwave{t o}mum}.
Unreleased stops will be most noticeable at major pauses, as in \N{oel
omum se\uwave{t}.}

\subsubsection{} The \N{r} is an alveolar flap.  The \N{l} is clear
and front, as in ``leaf,'' not the velarized, ``dark-l'' of English
``call''.

\subsubsection{} Frommer devised a scientific orthography in which two
of the digraphs were written as a single letter, \N{c} for \N{ts} and
\N{g} for \N{ng}.  The digraph system was easier for the actors, but
it has been also used by Frommer in media interviews and in most of
his own email.  The scientific orthography is only seen in a few early
emails to and from Frommer.  \label{l-and-s:cg}

\subsubsection{} Because plain stops can be used as syllable codas,
the more common ejective notation, \N{p'}, is too ambiguous:
\N{tsap'alute} is not *\N{tsapxalute}.
\LNWiki{21/12/2009}{http://wiki.learnnavi.org/index.php/Canon\%23Extracts_from_various_emails}

\newpage
\subsection{Vowels}

\begin{center}
\begin{tabular}{ccccc}
\N{i} [i], \N{ì} [{\footnotesize I}]  & & & & \N{u} [u],[ʊ] \\
 & \N{e} [ɛ] & & & \N{o} [o] \\
 & & \N{ä} [æ] &  \N{a} [ɑ] \\
\end{tabular}
\end{center}

\subsubsection{} The phoneme \N{u} is always [u] in open syllables,
and may be either [u] or [ʊ] in closed syllables.  \N{Lu} is always
pronounced [lu], while \N{tsun} may be either [tsun] or [tsʊn].
\LNWiki{20/5/2010}{http://wiki.learnnavi.org/index.php/Canon/2010/March-June\%23The_Dual_sounds_of_.22u.22}

\subsubsection{} The diphthongs are \N{aw}, \N{ay}, \N{ew} and \N{ey}.
Only in diphthongs will \N{w} or \N{y} be seen at the end of a
syllable (\N{new}) or before a final consonant (\N{hawng}).

\subsection{Pseudovowels} The pseudovowel \N{rr} is a syllabic,
trilled [{\gplus r̩ː}], and \N{ll} is a syllabic [{\gplus l̩ː}].

\subsection{Syllable Structure}
 Na'vi has a strict but straightforward syllable structure.

\begin{itemize*}
  \item A syllable is permitted to have no onset consonant (i.e., it
    may start with a vowel).
  \item A syllable is permitted to have no coda consonant (i.e., it
    may end with a vowel).
  \item Any consonant may start a syllable.
  \item A consonant cluster of \N{f s ts} $+$ \N{p, t, k, px, tx, kx,
    m, n, ng, r, l, w, y} may start a syllable (e.g., \N{tslam}, \N{ftu}).
  \item \N{Px tx kx p t k ' m n l r ng} may occur in syllable-final position.
  \item \N{Ts f s h v z w y} may \textit{not} occur in syllable-final position.
  \item There are no consonant clusters in syllable-final position.
  \item \label{l-and-s:pseudo-no-null} A syllable with a pseudovowel
    must start with a consonant or consonant cluster and must not have
    a final consonant; this plays a role in lenition
    (\horenref{l-and-s:lenition:pseudovowel}) and the declension of nouns
    (\horenref{morph:decl:pseudovowel}).
\end{itemize*}

\subsubsection{} Since a syllable may lack a consonant onset or coda,
it is not unusual to see several vowels next to each other in a word.
In that case each vowel is a syllable, \N{muiä} [mu.i.æ], \N{ioang}
[i.o.ɑŋ].

\subsubsection{} In general, the sequence VCV will be syllabified V.CV
rather than VC.V, so \N{tsenge} is [tsɛ.ŋɛ] not *[tsɛŋ.ɛ].
Onomatopoeia may override this, as in \N{kxang\-ang\-ang} [k'ɑŋ.ɑŋ.ɑŋ],
where the echo effect is desired.

\subsubsection{} There are no long vowels in Na'vi, meaning identical
vowels will not occur next to each other (but see
\horenref{l-and-s:contract}).

\subsubsection{} Double consonants do not occur in root words, but
may occur at morpheme boundaries, for example in derivations,
\N{tsukkäteng} $<$ \N{tsuk-} $+$ \N{käteng}, or with enclitics
\N{Mo'atta} $<$ \N{Mo'at} $+$ \N{ta} (\horenref{l-and-s:stress:enclisis}).
% http://naviteri.org/2011/03/“receptive-ability”-and-hesitation/comment-page-1/#comment-604

\subsubsection{} As is usual in most Human languages, some
interjections break the rules, such as \N{oìsss}, a sound for anger, or
\N{saa}, a threat cry.


\subsection{Stress Accent}
Every Na'vi word has at least one stress accent, which is not
predictable.  In a very few situations otherwise identical words may
differ only by accent, such as \N{\ACC{tu}te} \E{person}
vs. \N{tu\ACC{te}} \E{woman}.

\subsubsection{} For this word alone, \E{woman}, an accent may be
written in normal Na'vi to indicate the accent, \N{tuté}.
\index{tuté@\textbf{tuté}}

\subsubsection{} Some word creation processes may cause accent shifts
(\horenref{lingop:prefix:ke}, \horenref{lingop:suffix:gender}).

\subsubsection{} All adpositions as well as a few conjunctions and
particles may be enclitic.  They give up their own stress accent and
effectively become part of the word to which they are attached, and
are written so, \N{\ACC{tsa}ne} ($<$ \N{tsaw} $+$ \N{ne}),
\N{ho\ACC{ren}\-ti\-sì} ($<$ \N{ho\ACC{ren}ti} $+$ \N{sì}).
\label{l-and-s:stress:enclisis}\index{enclitics}

\subsubsection{} Though a noun compound is written as a single word,
the individual parts of that compound may each retain their original
accent, as in \N{ti\ACC{re}a\ACC{fya}'o} \E{spirit path}.
\index{compound word!accent}

%\subsubsection{} Word stress is a property of stem words.  No matter
%how many affixes a root word takes, no secondary accents develop.

\subsection{Spoken Alphabet}
Except for \N{tìftang}, the glottal stop, the names of the phonemes
encode information about how the sound is used.  They also have
unusual capitalization when written out: \index{alphabet!spoken}

\begin{center}\small
\begin{tabular}{lll}
\N{tìftang} & \N{Ì} & \N{ReR} \\
\N{A}  & \N{KeK}   & \N{'Rr} \\
\N{AW} & \N{KxeKx} & \N{Sä} \\
\N{AY} & \N{LeL}   & \N{TeT} \\
\N{Ä}  & \N{'Ll}   & \N{TxeTx} \\
\N{E}  & \N{MeM}   & \N{Tsä} \\
\N{EW} & \N{NeN}   & \N{U} \\
\N{EY} & \N{NgeNg} & \N{Vä} \\
\N{Fä} & \N{O}     & \N{Wä} \\
\N{Hä} & \N{PeP}   & \N{Yä} \\
\N{I}  & \N{PxePx} & \N{Zä} \\
\end{tabular}
\end{center}

\subsubsection{} Vowels and diphthongs are simply pronounced and
spelled as themselves.  The pseudovowels take a leading glottal stop,
since they require a consonant onset (\horenref{l-and-s:pseudo-no-null}).

\subsubsection{} The name for consonants which cannot end a syllable
are formed by adding \N{ä}, as in \N{Tsä}.  Those which can end a
syllable use the vowel \N{e} and repeat the consonant at the end of
the name, \N{PeP}.


\section{Lenition}
\noindent Certain grammatical processes cause changes in the first
consonant of a word.  This change is called ``lenition.''  Only eight
consonants undergo lenition.\index{lenition}\label{l-and-s:lenition}
\LanguageLog

\begin{center}
\begin{tabular}{lll}
Consonant & Lenition & Example \\
\N{px, tx, kx} & \N{p, t, k} & \N{\uwave{tx}ep} but \N{mì \uwave{t}ep} \\
\N{p, t, k} & \N{f, s, h} & \N{\uwave{k}elku} but \N{ro \uwave{h}elku} \\
\N{ts} & \N{s} & \N{\uwave{ts}mukan} but \N{ay\uwave{s}mukan} \\
\N{’} & disappears & \N{’eylan} but \N{fpi eylan} \\
\end{tabular}
\end{center}

\subsection{Glottal Stop} The glottal stop is not lenited when it is
followed by a pseudovowel (\N{mì 'Rrta} not *\N{mì Rrta}).
\index{glottal stop!lenition}\label{l-and-s:lenition:pseudovowel}

\subsection{Adpositions} A few adpositions cause lenition when they
precede a word: \N{fpi}, \N{ìlä}, \N{mì}, \N{nuä}, \N{ro}, \N{sko},
\N{sre} (and derived \N{lisre} and \N{pxisre}), \N{wä}. When suffixed
they do not cause lenition in either the word they are attached to or
to the following word.
\index{fpi@\textbf{fpi}!lenition}\index{ilä@\textbf{ìlä}!lenition}
\index{miì@\textbf{mì}!lenition}\index{ro@\textbf{ro}!lenition}
\index{sre@\textbf{sre}!lenition}\index{pxisre@\textbf{pxisre}!lenition}
\index{waä@\textbf{wä}!lenition}\index{nuaä@\textbf{nuä}!lenition}
\index{sko@\textbf{sko}!lenition}
\index{lenition!adpositions}\index{adpositions!lenition}

\subsection{Number Prefixes} Prefixes which cause lenition are
indicated with a plus sign, rather than the usual dash, as in \N{ay+},
the leniting plural prefix. \index{lenition!number prefixes}

\subsection{Question Prenoun} When used as a prefix, the question
prenoun \N{pe+} causes lenition (\horenref{morph:pre:pe}).

\subsection{Numbers} Suffixed, dependent forms of the numbers are
lenited (\horenref{numbers:dependent}). \index{lenition!numbers}

\subsection{Proper Nouns} Proper nouns still undergo lenition. \N{oe kelku si mì Helutral} \E{I live in Hometree}
\index{lenition!proper nouns}\NTeri{10/28/2010}{http://naviteri.org/2010/09/getting-to-know-you-part-2/}

\section{Morphophonology}

\subsection{Vowel Contraction} Since identical vowels may not occur
next to each other, a few grammatical processes involve a doubled
vowel reducing to just one.\index{vowel!contraction}\label{l-and-s:contract}

\subsubsection{} The adjective morpheme \N{-a-} disappears when
attached to an \N{a} at the start or end of an adjective, as in
\N{apxa tute} not *\N{apxaa tute}.
\index{-a-@\textbf{-a-}!with \textbf{a} in an adjective}
\index{adjective!contraction}

\subsubsection{} When the dual and trial prefixes leave a sequence of
two \N{e}s, as in \N{me} $+$ \N{'eveng} $>$ *\N{meeveng} (note
lenition), the two vowels contract to just one, \N{meveng}.
\label{l-and-s:phonotactics:nsc} \index{dual!contraction}
\index{trial!contraction}
\LNWiki{20/1/2010}{http://wiki.learnnavi.org/index.php/Canon\%23Extracts_from_various_emails}

\subsubsection{} When the prenoun prefixes end in the same vowel the
following word starts with, they reduce to one, as in \N{tsatan} $<$
\N{tsa-} $+$ \N{atan}, \N{fìlva} $<$ \N{fì-} $+$ \N{ìlva}
(\horenref{morph:prenoun:contraction}).\footnote{The glottal stop is a
consonant, so \N{fì'ìheyu} from \N{fì-} $+$ \N{'ìheyu}.}
\label{l-and-s:phonotactics:precontract}\index{prenoun!contraction}
\LNWiki{18/5/2011}{http://wiki.learnnavi.org/index.php/Canon/2011/April-December\%23Kawtseng.2C_tsapo_and_prefixes}

\subsubsection{} Contraction does not occur for indefinite \N{-o} or
enclitic adpositions.  When two identical vowels occur next to each
other, they are written with a hyphen between them, \N{fya'o-o}
\E{some way,} \N{zekwä-äo} \E{under a finger}.\footnote{Though Na'vi
does not technically have long vowels, the effect of long vowels
occurs in this situation.  Take care to pronounce both \N{ä} in a word
such as \N{zekwä-äo}.}\index{vowel!contraction!inhibited}
% http://wiki.learnnavi.org/index.php/Canon/2010/UltxaAyharyuä#Phonological_Questions

\subsection{Pseudovowel Contraction} Due to the shape of the aspect
infixes, \N{\INF{er}} and \N{\INF{ol}}, it is possible for the
pseudovowels to occur immediately after their consonantal counterpart,
as in \N{*p\INF{ol}ll\ACC{txe}}.  When this happens in an unstressed
syllable, the pseudovowel disappears, \N{pol\ACC{txe}}.  In a
stressed syllable, the infix disappears, \N{*\ACC{f}\INF{er}\ACC{rr}fen} $>$
\N{\ACC{frr}fen}.  Pseudovowels in monosyllables behave as though
unaccented, \N{vol} from \N{*v\INF{ol}ll}. \index{pseudovowel!contraction}
\LNWiki{23/3/2010}{http://wiki.learnnavi.org/index.php/Canon/2010/March-June\%23Misc_Answers}
\NTeri{19/6/2012}{http://naviteri.org/2012/06/spring-vocabulary-part-3/}

\subsection{Affect Infix Epenthesis} When the positive affect infix
\N{\INF{ei}} is followed by the vowel \N{i}, \N{ì} or a pseudovowel, a
\N{y} is inserted, \N{seiyi} $<$ \N{*s\INF{ei}i}, \N{veykrreiyìn} $<$
\N{*veykrr\INF{ei}ìn}; \N{v\INF{ei}yll} $<$ \N{*veill}.
\label{l-and-s:eiy-epenth}
\NTeri{19/6/2012}{http://naviteri.org/2012/06/spring-vocabulary-part-3/}

\subsection{Nasal Assimilation} In many compounds as well as in some
idioms, final nasals assimilate to the position of the following
word, as in \N{lumpe} as a variant of \N{pelun}.  Such assimilation is
not always written, which may make the etymology of a word clearer, as
in \N{zenke} instead of \N{*zengke}, from \N{zene ke}, or in the several
idioms with the verb \N{tìng} \E{give}, \N{tìng mikyun} being pronounced
\N{tìm mikyun}. \index{nasal assimilation} \label{l-and-s:nasalassim}

\subsection{Vowel Harmony} Na'vi has two instances of optional
regressive vowel harmony in verb infixes.\index{vowel!harmony}

\subsubsection{} The subjunctive future infix, \N{\INF{iyev}}, most
frequently appears as \N{\INF{ìyev}}, with backing of the first vowel.

\subsubsection{} The vowel of the negative attitude infix,
\N{\INF{äng}}, may be raised if it is immedately followed by the vowel
\N{i}, becoming \N{\INF{eng}}, as in \N{tsap'alute \uwave{sengi}
oe}.\label{l-and-s:eng}
\Ultxa{2/10/2010}{http://wiki.learnnavi.org/index.php/Canon/2010/UltxaAyharyu\%C3\%A4\%23.C3.A4ng.2Feng}

\subsection{Elision} In rapid speech final \N{-e} is frequently elided
when the following word starts in a vowel.  \Npawl{Kìyevam$\not$e
ult$\not$e Eywa ngahu}.  This is not indicated in writing.\index{elision}
\QUAESTIO{But not monosyllables?  \N{ke}? \N{sre}?}

\subsubsection{} The vowel \N{ì} in \N{mì}, \N{sì} and the adverb
prefix \N{nì-} drops before the plural prefix \N{ay+}, though there is
no change in writing.  So, \N{nìayfo} \E{like them} is pronounced as
\N{nayfo}. \label{l-and-s:elision-i}
\index{miì@\textbf{mì}!elision with plural}
\index{siì@\textbf{sì}!elision with plural}
\index{niì-@\textbf{nì-}!elision with plural}
\NTeri{1/7/2010}{http://naviteri.org/2010/07/thoughts-on-ambiguity/}

\subsubsection{} The vowel in \N{nì-} will usually elide before a
stressed \N{e}, as in \N{nì-} + \N{etrìp} > \N{netrìp}. If the \N{e}
is unstressed, it will usually, though not always, elide, \N{nì-} +
\N{eyawr} > \N{nìyawr}. One exception: \N{nìean} instead of the
expected \N{*nìan}.
\index{niì-@\textbf{nì-}!elision before e}
\LNForum{9/8/2017}{https://forum.learnnavi.org/language-updates/if-ni-will-attached-at-e/}

\section{Orthographic Conventions}
\noindent Na'vi in general follows the spelling, punctuation and
capitalization habits of English, but there are a few differences.

\subsection{Proper Names} When taking lexical prefixes
(\horenref{lingop:affixes}), proper names retain their original
capitalization, as in \N{lì'fya le\uwave{Na'vi}}.

\subsection{Quotation} Direct quotes are not punctuated with quotation
marks in Na'vi.  Instead it relies on the quotation particles
\N{san\dots sìk} (see \horenref{syn:direct-quote}).
\index{quotation!punctuation}

\subsection{Etymological Spelling} In addition to the occasional
spelling of nasals to reflect etymo\-logy (\horenref{l-and-s:nasalassim}),
there are a few grammatical processes which result in spelling that
reflects the grammar more than the pronunciation.

\subsubsection{} The first person pronoun root \N{oe}, though
pronounced \N{we} when taking a suffix, retains the original spelling
(\horenref{morph:pron:oe-we}).

\subsubsection{} Before words starting with \N{y} the plural prefix
\N{ay+} is unchanged, \N{ayyerik}.
\LNWiki{18/4/2010}{http://wiki.learnnavi.org/index.php/Canon/2010/March-June\%23ay.2Byerik}

\subsection{Attributive Phrase Hyphenation} Certain short attributive
phrases are written with hyphens joining the elements.

\subsubsection{} Attributive phrases of color using \N{na} \E{like}
are hyphenated, \N{fìsyulang aean-na-ta'leng} \E{this skin-blue
flower} (\horenref{syn:attr:na}).

\subsubsection{} Participles of \N{si} construction verbs are also
hyphenated, \N{srung-susia tute} \E{a helping person}
(\horenref{syn:participle:si-const}).
