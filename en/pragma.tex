\nchapter{Pragmatics and Discourse}

\noindent In previous chapters I have discussed sounds, words and
sentences in Na'vi.  Much of that discussion took the form of rules.
This chapter is devoted to language one step higher than even the
sentence --- conversation, narrative and the contexts in which
language takes place, what linguists group together under the name
pragmatics.  Simple rules are harder to come by here, so the
discussion necessarily takes a slightly different structure.


\section{Constituent Order}

\subsection{Free Word Order} Na'vi has been described has having free
word order.  This is a little misleading since that phrase means
something quite specific to linguists.  Rather, Na'vi has free
constituent order.\footnote{A \textit{constituent} is a slightly
bigger building block than the word, but smaller than a sentence.  A
constituent is a group of words that function as a single
grammatical unit.  For example, in the sentence, ``the big bad wolf
ate Little Red Riding Hood's grandmother,'' the phrase ``the big bad
wolf'' is one consituent acting as the subject, the verb ``ate''
stands on its own and ``Little Red Riding Hood's grandmother'' is
the direct object constituent.  Sometimes a constituent can be a
single word (``he ate her'' --- each word a constituent) and
sometimes they can be quite a lot more complex.}  Within
constituents, word order may be quite constrained.  You cannot stick
part of one constituent into the middle of another.  For example, in
\N{ayoel tarmaron tsawla yerikit} \E{we were hunting a large hexapede},
I cannot break apart the direct object constituent \N{tsawla yerikit}
and produce things like \N{*tarmaron tsawla ayoel yerikit} or \N{*ayoel
tsawla tarmaron yerikit}.\index{word order}

\subsubsection{} In complex constituents it is possible for a genitive
to be separated from its noun by a relative clause, \N{ngeyä teri
faytele a aysänumeri} \E{your instructions about these matters}.

\subsection{SOV, SVO, VSO} Many human languages can conveniently be
categorized based on their default word order for subject, verb and
direct object, usually shortened to just S, V and O.  English is
mostly an SVO language, Japanese is SOV.  Free word order languages are
not easily categorized into this system, though some do have
tendencies worth noting.  Looking at Frommer's Na'vi, and only
counting sentences with all three constituents, we can say that the
three main word orders are SVO, SOV and VSO, with a very slight
preference for VSO.  Other orders, such as OVS and OSV, are much
rarer.\footnote{This is based off two of the larger pieces of
connected Na'vi text Frommer has produced, his first blog post and his
message on the MaSempul.org web site.
\begin{center}
\begin{tabular}{llll}
\N{Order} & \N{Blog} & \N{Ma Sempul} & Total \\
\hline
SVO & 2 & 3 & 5\\
SOV & 4 & 1 & 5\\
VSO & 5 & 2 & 7\\
OVS & 0 & 1 & 1\\
OSV & 0 & 2 & 2\\
\end{tabular}
\end{center}}

\subsection{Word Order Effects} \index{word order!effects} Changes in
word order can sometimes motivate changes in grammar. \label{pragma:woe}

\subsubsection{} \label{pragma:word-order-effects:modals} If a
sentence is ordered such that a modal and its controlled, transitive
verb are contiguous, and the subject and direct object are contiguous,
the modal and verb combination may be reanalyzed as a single
transitive verb.  For example, \Npawl{oe teylut new yivom} \E{I want
to eat teylu} has the expected, correct case use, with the subject
of the modal in the subjective case, the direct object in the
patientive case (\horenref{syn:modals}).  However, in a few word
orders the subject may be put in the patientive case.  In decreasing
order of acceptability:

\begin{center}
\begin{tabular}{lr}
\N{\uwave{Oel} teylut new yivom.} & widely acceptable\footnotemark \\
\N{Teylut \uwave{oel} new yivom.} & about 50\% acceptable \\
\N{New yivom teylut \uwave{oel}.} & about 30\% acceptable \\
\N{*New yivom oel teylut.} & completely unacceptable 
\end{tabular}
\end{center}
\footnotetext[\value{footnote}]{According to Frommer's blog, ``...in
all but the most formal situations.''}

\subsection{Focus} \index{focus}\index{word order!focus}\index{punch}
Since free word order languages do not use word order for syntax, they
are free to use it to indicate other things, such as style, emphasis
and focus.  The only thing Frommer has told us for certain about Na'vi
word order is, ``the end of the sentence is where the `punch' comes.''
We can take this to mean that if you wish to emphasize a constituent,
put it at the end of the clause.  Notice in particular how Frommer
translated this sentence:

\begin{quotation}
\noindent\Npawl{Fkxilet a tsawfa poe ioi säpalmi ngolop \uwave{Va'rul}.}\\
\indent\E{\uwave{Va'ru} is the one who created the necklace she was wearing.}
\end{quotation}
% http://naviteri.org/2011/08/new-vocabulary-clothing/

\noindent The focused, salient part of the answer to a question is
similarly moved to the end of the clause:

\begin{quotation}
\noindent\Npawl{Spaw oel futa Mo’atìl tsole’a Neytirit.}\\
\indent\E{I believe Mo'at saw Neytiri.}\\

\noindent\Npawl{Kehe. Tsole'a Neytirit Eytukanìl.}\\
\indent\E{No. The one who saw Neytiri was Eytukan.}
\end{quotation}

\noindent In English such focus can also be handled by emphasizing a
particular word with stress, \E{\uline{Eytukan} saw Neytiri}.
\NTeri{19/3/2011}{http://naviteri.org/2011/03/word-order-and-case-marking-with-modals/}

% Add a note about the keng example in:  ???   2018jan02
% http://naviteri.org/2012/02/trr-asawnung-lefpom-happy-leap-day/

\subsection{The English Passive} Although generations of English
teachers have convinced many people that the passive voice is weak and
flimsy, it is in fact simply one tool English uses to organize
information clearly for listeners. \index{passive!with word order}
The passive lets us bring the patient of some action to prominence by
moving it to the head of the sentence.  If we say, ``the nun was run
over by a car'' we're communicating to our listeners that the nun is
the most salient part of the sentence, and that the exact vehicle is
less a concern.\footnote{In English, we can even omit the agent of a
passive altogether, ``the nun was run over.''}  Na'vi does not have
a passive voice, but Frommer has suggested the word order OSV as one
way to communicate the same effect (but see also \N{fko},
\horenref{syn:prn:fko}).


% \QUAESTIO{Thetic vs. categorical statements.}


\section{Topic-Comment}
\label{pragma:topic-comment}\index{case!topical}

\noindent The topic-comment construction is conceptually
straightforward: the ``topic'' announces what the rest of the sentence
relates to, and the comment makes some statement relating to that
topic.  While plenty of Human languages organize discourse heavily
around topic-comment structure, English is not one of them.  This can
make it difficult to present decent translations of topic-comment
structures that are both true to the meaning of the original but which
also make clear the structure under discussion.  In this section I'll
often use prepositional phrases with ``as for'' and ``concerning'' for
all examples, but this is a clunky work-around, used only for clarity.

\subsection{Topical Case} In Na'vi only nouns, noun phrases and
pronouns can be topics.  These are marked with the topical case
(\N{-ri, -ìri}).  More complex topics can be created with nominalized
clauses (\horenref{syn:clause-nom}).

\subsection{Topical Role} The topic can be especially confusing for
those unfamiliar with it because almost any syntactic role in a
sentence may be pulled out of the sentence to be the topic.  One
idiomatic use is marking inalienable possession
(\horenref{syn:topical:poss}). But you can use the topical where
English would use a direct object:

\begin{quotation}
\noindent\Npawl{Fayupxare layu aysngä’iyufpi, fte \uwave{lì'fyari
awngeyä} fo tsìyevun nìftue nìltsansì nivume.}

\medskip
\noindent\E{These messages will be for beginners so that they can
learn \uwave{our language} easily and well.} 
\end{quotation}

\noindent But the relationship between the topic may not match a
strict syntactic role, too:

\begin{quotation}
\noindent\Npawl{Ma oeyä eylan, \uwave{faysänumviri} rutxe fì’ut tslivam: \dots}\\
\indent\E{My friends, \uwave{concerning these lessons,} please
  understand this: \dots}

\medskip
\noindent\Npawl{\uwave{Ayngeyä sìpawmìri} kop fmayi fìtsenge tivìng
  sì’eyngit.}\\
\indent\E{\uwave{As for your questions,} (I) will also try to give answers
  (for them) here.}
\end{quotation}

\subsubsection{} A topic may introduce a complex sentence, coming
before even a leading conjunction,

\begin{quotation}
\noindent\Npawl{\uwave{Fori} mawkrra fa renten ioi säpoli holum.}\\
\indent\E{After they put on their goggles, they left.}
\end{quotation}
% http://naviteri.org/2011/08/new-vocabulary-clothing/

\subsubsection{} Similarly, a topic may apply for multiple comments, 

\begin{quotation}
\noindent\Npawl{\uwave{Poeri} uniltìrantokxit tarmok a krr, lam stum
nìayfo, slä lu 'a'awa tìketeng --- natkenong, \uwave{tsyokxìri} ke lu
zekwä atsìng ki amrr.}

\medskip
\noindent{\uwave{As for her}, when \uwave{she} inhabited an avatar,
\uwave{she} was almost like them, but there were a few differences
--- for example, \uwave{as for her hand} there were not four fingers
but five.}
\end{quotation}


\subsection{Using the Topical} Each human language has its own rules
and tendencies about when the topical should be used.  At this stage
it is a little difficult to set rules for this, but a few tendencies
can be drawn from what we have seen so far.  First, so far Frommer has
not used topic-comment constructions nearly as often as it is used in
Chinese or Japanese (both topic-prominent languages, though each in
their own way).  Second, Frommer does not use the topical to introduce
new matters for discussion, rather topics refer to current matters, or
matters which are readily inferable from the conversation.

English uses the definite article, \E{the,} to mark information that
has already been introduced into discourse, as well as information
that can be assumed or deduced from the conversation.  For example, if
I say, ``I wanted to see \E{Avatar,} but the line was too long,'' I
can use the definite article with \E{line} not because we've been
talking about lines, but because standing in line is something we're
used to when seeing a popular film.  In comments on a recent blog
post\footnote{\href{http://naviteri.org/2010/08/20/}{A
Na'vi Alphabet}, August 20, 2010} Frommer says, \index{case!topical!as definite}

\begin{quote}But if the message is indefinite, the topical case
doesn’t work as well, since topics are usually definite. So
\N{'upxareri ngaru pamrel soli trram} can certainly mean \E{I wrote
you THE message yesterday.} Can it also mean \E{I wrote you A
message yesterday}? Since there are no articles per se in Na’vi
and nouns can be either definite or indefinite, I guess it
could. But something about it rubs me the wrong way.
\end{quote}

\noindent It seems best to avoid indefinite topicals for now.


\section{Register}

\subsection{Formal Register} Na'vi has two main ways to mark
ceremonial or formal speech: with special pronouns
(\horenref{morph:hon-pron}) and with the verb affect infix
\N{\INF{uy}} (\horenref{morph:verb:2nd-pos}).\index{register!formal}

\subsubsection{} The formal pronouns may be used in close succession,
\N{Sìfmetokit emzola'u \uwave{ohel}. Ätxäle si tsnì livu \uwave{oheru}
Uniltaron} \E{I have passed the tests. I respectfully request the
Dream Hunt.}

\subsubsection{} Like the tense and aspect markers, it is not
necessary to repeat the infix \N{\ACC{uy}} once a formality context
has been established.

\subsubsection{} Solemnity or sincerity of a statement may be shown by
using both pronoun and verb formality marking, \Npawl{faysulfätuä
tìkangkem \uwave{oheru} meuia \uwave{luyu} nìngay} \E{the work of these
experts is truly an honor for me}.


\subsection{Poetic Register}

\subsubsection{} In prose the topical will come first in its clause or
immediately after a conjunction (\horenref{syn!topical!word-order}).
In verse, it may move deeper into the clause structure,
\Npawl{\uwave{pxan} livu txo nì'aw oe \uwave{ngari} / tsakrr nga
Na'viru yomtìyìng} \E{only if I am \uwave{worthy of you} will you feed
the people}.

\subsubsection{} In normal prose, when an adposition comes before the
noun or noun phrase, any genitive must also come after the adposition,
as in \N{fa oeyä tsyokx} or \N{fa tsyokx oeyä} \E{with my hand}.  In
poetry, the genitive may also come before the adposition, \N{oeyä fa
tsyokx}. 
\LNWiki{17/3/2012}{http://wiki.learnnavi.org/index.php/Canon/2012/January-June\%23A_poetic_license_and_a_note_on_adposition_position}

\subsubsection{} In day-to-day speech a modal verb must come before
its controlled verb (\horenref{syn:modal-syntax}).  In poetic or
ceremonial language, the modal may follow.
\NTeri{3/19/2011}{http://naviteri.org/2011/03/word-order-and-case-marking-with-modals/}


\subsection{Colloquial Register} The colloquial register presents
itself mostly in simplified grammar or abbreviated expression.
\index{register!colloquial}

\subsubsection{} Verbs of cognition may introduce a subclause without
any conjunction.

\begin{quotation}
\noindent\E{I believe it was a mistake for him to have gone.}\\
\noindent\Npawl{\uwave{Spängaw oel futa} fwa po kolä lu kxeyey.}\\
\noindent Colloquial: \Npawl{\uwave{Spaw oe}, fwa po kolä längu kxeyey.}
\end{quotation}

\subsubsection{} In casual conversation the reflexive perfective of
\N{si}-construction verbs, \N{säpo\ACC{li}}, is often pronounced
\N{spo\ACC{li}}.
\NTeri{3/8/2011}{http://naviteri.org/2011/08/new-vocabulary-clothing/}


\subsection{Slang} At the request of the community, Paul Frommer approved
some constructions deviating from standard grammar that could be used
as `slang' among the Na'vi in informal contexts, or as a way to have
fun with the language.
\index{register!slang}\index{slang}

\subsubsection{}The attitude infixes \N{\INF{ei}} and \N{\INF{äng}},
normally used in verbs, may be inserted into \N{srane} \E{yes}
and \N{kehe} \E{no} instead of using adverbs to express a mood.
Following the common pattern in verbs, the attitude infixes go into
the second syllable (\horenref{morph:verb:2nd-pos}),
\N{sran\INF{äng}e}, \N{keh\INF{ei}e}. 
\LNForum{19/04/2020}{https://forum.learnnavi.org/language-updates/regarding-some-memetic-uses-of-na'vi/}

\subsubsection{}Proper names may form a compound verb with \N{si} to
express \E{do as X} as an extension to the current formation method of
these verbs.
\LNForum{19/04/2020}{https://forum.learnnavi.org/language-updates/regarding-some-memetic-uses-of-na'vi/}


\subsection{Clipped Register} In military settings certain features of
grammar can be modified or omitted for brevity.
\index{register!military}\index{register!clipped}

\subsubsection{} In noun phrase utterances, participles may go with
their noun without using the attri\-bu\-tive affix \N{-a-}
(\horenref{syn:part:attr}), \N{tìkan tawnatep} \E{target lost} (from
the video game).
\LNWiki{21/5/2010}{http://wiki.learnnavi.org/index.php/Canon/2010/March-June\%23Losing_and_registers}

\subsubsection{} Some pronoun genitives lose the final \N{-ä}, see
\horenref{morph:pron:gen-clipped}.  This may be used casually, in
non-military situations, among friends or close acquaintances.
\LNWiki{21/5/2010}{http://wiki.learnnavi.org/index.php/Canon/2010/March-June\%23Losing_and_registers}

