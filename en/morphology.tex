\nchapter{Morphology}

\section{The Noun}

\subsection{Cases} The Na'vi case endings change depending on whether
the word ends in a consonant, a vowel or a diphthong.\footnote{The
case names Frommer uses reflect the terminology used by Bernard
Comrie in his writings on ergative languages.  In most linguistic
writing Frommer's ``subjective'' is called the intransitive, the
``agentive'' is the ergative and the ``patientive'' the accusative.}
\index{noun!declension}

\begin{center}
\begin{tabular}{lccc}
 & Vowel  & Consonant \& Pseudovowel & Diphthong \\
\hline
Subjective & --- & --- & --- \\
Agentive & \N{-l} & \N{-ìl} & \N{-ìl} \\
Patientive & \N{-t}, \N{-ti} & \N{-it}, \N{-ti} & \N{-ti}, \N{-it} (\N{-ay-t, -ey-t}) \\
Dative & \N{-r}, \N{-ru} & \N{-ur} & \N{-ru}, \N{-ur} (\N{-aw-r, -ew-r}) \\
Genitive & \N{-yä}, \N{-o-ä}, \N{-u-ä} & \N{-ä} & \N{-ä} \\
Topical  & \N{-ri} & \N{-ìri} & \N{-ri}  \\
\end{tabular}\end{center}

\noindent\LNWiki{24/3/2010}{http://wiki.learnnavi.org/index.php/Canon/2010/March-June\%23Declension_with_Diphthongs_and_Deixis}

% For gen. with i: http://forum.learnnavi.org/language-updates/genitive-case-refinement-declension-of-tsaw/msg150927/#msg150927

\subsubsection{} Note that words ending in the pseudo-vowels \N{ll}
and \N{rr} take the consonant endings: \N{trr-ä}, \N{'ewll-it}.
\index{pseudovowel!declension}\label{morph:decl:pseudovowel}

\subsubsection{} After the vowels \N{o} and \N{u} the genitive is just
\N{-ä}, but after all other vowels it is \N{-yä}.  So, \N{tsulfätuä}
from \N{tsulfätu}, but \N{Na'viyä} from \N{Na'vi} and \N{lì'fyayä}
from \N{lì'fya}.

\subsubsection{} Nouns in \N{-ia} have the genitive in \N{-iä}, as in
\N{soaiä} from \N{soaia}.
% http://naviteri.org/2011/05/some-miscellaneous-vocabulary/
% This used to be covered by a single example in following "case
% refinement" section link; later generalized.

\subsubsection{} In addition to several pronouns
(\horenref{morph:pron:irreg-gen}), there are a few nouns with
irregular genitives: \N{Omatikayaä} (from \N{Omatikaya}).
% http://forum.learnnavi.org/language-updates/genitive-case-refinement-declension-of-tsaw/msg150927/#msg150927

\subsubsection{} Due to the similarity in sound between \N{y} and
\N{i}, the patientive ending \N{-it} is simplified when suffixed to a
diphthong ending in \N{y}, as in \N{keyeyt} \E{errors} instead of
*\N{keyeyit}.  And due to similarity in sound between \N{w} and and
\N{u}, the same simplification happens to the dative \N{-ur}, as in
\N{'etnawr} \E{to/for a shoulder} instead of *\N{'etnawur}.
\NTeri{1/25/2013}{http://naviteri.org/2013/01/awvea-posti-zisita-amip-first-post-of-the-new-year/}

\subsubsection{} The variation between the long and short endings in
the patientive and dative appears to be largely a matter of style and
euphony.

\subsection{Indefinite -o} A noun may take the indefinite suffix
\N{-o}, ``one, some.''  Case endings follow the \N{-o}.
\index{-o@\textbf{-o}}\index{indefinite noun}
\LNWiki{14/3/2010}{http://wiki.learnnavi.org/index.php/Canon/2010/March-June\%23
A_Collection}
\NTeri{5/9/2011}{http://naviteri.org/2011/09/\%E2\%80\%9Cby-the-way-what-are-you-reading\%E2\%80\%9D/comment-page-1/\%23comment-1093}

\subsection{Number} Na'vi nouns and pronouns may be singular, dual,
trial or plural (four or more).  Number is indicated by prefixes, all
of which cause lenition.\index{dual}\index{trial}\index{plural}

\begin{center}
\begin{tabular}{lrl}
Dual & \N{me+} & \N{mefo} ($<$ \N{me+} $+$ \N{po}) \\
Trial & \N{pxe+} & \N{pxehilvan} ($<$ \N{pxe+} $+$ \N{kilvan}) \\
Plural & \N{ay+} & \N{ayswizaw} \\
\end{tabular}
\end{center}

\subsubsection{} The plural prefix \textit{only} may be dropped if
there is lenition.  The plural of \N{prrnen} is either \N{ayfrrnen} or
the short plural \N{frrnen} (but see
\horenref{syn:adp:short-plural}).\footnote{Exception: \N{'u} \E{thing}
does not take the short plural, always occuring as
\N{ayu}.\index{'u@\textbf{'u}!no short plural}}
The dual and trial prefix are never dropped this way.
\index{plural!short} \label{morph:short-plural}
\LanguageLog

\subsubsection{} In the dual and trial, if a word begins with \N{e} or
\N{’e}, the resulting \N{*ee} is simplified, so \N{me+} $+$ \N{’eveng}
is \N{meveng}.  See also \horenref{l-and-s:phonotactics:nsc}.


\section{The Pronoun}

\subsection{Animacy} Animals may be referred to with the animate
pronoun \N{po}, but bugs are not. The more important the
speaker's relationship to the animal, the more likely a form of \N{po}
is used.\index{animacy} \LNForum{25/2/2017}{https://forum.learnnavi.org/language-updates/about-fmetok-and-pofo/}

\subsection{The Basic Pronouns}
The pronouns take the same case endings as nouns.
\begin{center}
\begin{tabular}{rllll}
Person      & Singular & Dual & Trial & Plural \\ 
\hline
1st exclusive   & \N{\ACC{o}e}  & \N{m\ACC{o}e}  & \N{px\ACC{o}e}   & \N{ay\ACC{o}e} \\
1st inclusive   & —      & \N{o\ACC{e}ng} & \N{px\ACC{o}eng} & \N{ayo\ACC{e}ng}, \N{aw\ACC{nga}} \\
2nd         & \N{nga} & \N{me\ACC{nga}} & \N{pxe\ACC{nga}} & \N{ay\ACC{nga}} \\
3rd animate & \N{po}  & \N{me\ACC{fo}} & \N{pxe\ACC{fo}}  & \N{ay\ACC{fo}, fo} \\
3rd inanimate   & \N{\ACC{tsa}'u}, \N{tsaw} & \N{me\ACC{sa}'u} & \N{pxe\ACC{sa}'u} & \N{ay\ACC{sa}'u, sa'u} \\
reflexive & \N{sno} & — & — & — \\
indeterminate & \N{fko} & — & — & — \\
\end{tabular}
\end{center}

\subsubsection{} In everyday speech, when the first person root \N{oe}
does not occur at the end of the word, its pronunciation changes to
\N{we}, as in \N{oel} pronounced \N{wel}, \N{oeru} as \N{weru}.
However, this pronunciation does not happen to the dual and trial
forms, \N{moe} and \N{pxoe}, which would result in illegal consonant
clusters at the start of a word, such as *\N{mwel}.  This
pronunciation is indicated with the accenting underline on the
\N{e}. \label{morph:pron:oe-we}

\subsubsection{}The non-singular first person pronouns are either exclusive
(excluding the person addressed) or inclusive (including the person
addressed).  The inclusive ending, \N{-ng}, is from \N{nga}, which
reappears in full when a case ending is added.  The agentive of
\N{oeng} is \N{oengal}, not \N{*oengìl}.

\subsubsection{} \N{Ayoeng} has the short form \N{aw\uline{nga}}.
Both may be used freely with any case ending, though \N{awnga} is more
common.\index{ayoeng@\textbf{ayoeng}}\index{awnga@\textbf{awnga}}

\subsubsection{} The third person animate \N{po} does not distinguish
gender --- it will do for ``he'' or ``she'' in English.  However,
gendered forms do exist, \N{po\ACC{an}} \E{he} and \N{po\ACC{e}}
\E{she}, which are declined regularly, though they do not have plural
forms.  See \horenref{syn:pron:gender} for their use.
\label{morph:pron:gender}

\subsubsection{} Several pronouns have irregular genitives with vowel
changes,

\begin{center}
\begin{tabular}{cc}
Subjective & Genitive \\
\hline
\N{fko} & \N{fkeyä} \\
\N{nga} & \N{ngeyä} \\
\N{po} & \N{peyä} \\
\N{sno} & \N{sneyä} \\
\N{tsa'u} & \N{tseyä}
\end{tabular}
\end{center}

\noindent This vowel change occurs in all numbers, \N{feyä} $<$
\N{fo}, and in the first person inclusives, \N{awngeyä} $<$ \N{awnga}.
\index{nga@\textbf{nga}!genitive \textbf{ngeyä}}\label{morph:pron:irreg-gen}
\index{fko@\textbf{fko}!genitive \textbf{fkeyä}}
\index{po@\textbf{po}!genitive \textbf{peyä}}
\index{sno@\textbf{sno}!genitive \textbf{sneyä}}
\index{awnga@\textbf{awnga}!genitive \textbf{awngeyä}}

\subsubsection{} In informal and clipped military speech the final
\N{ä} may drop from the genitive of pronouns, \N{ngey
'upxaret}.\label{morph:pron:gen-clipped} \index{genitive!shortened
in pronouns}\index{pronouns!short genitive}

\subsubsection{} The third person inanimate, \N{tsa'u}, is simply the
demonstrative pronoun ``that,'' and has the genitive in \N{tseyä}.
In informal, rapid speech it may take the form \N{tsaw}, which
may be used with postpositions (\N{tsawfa}), but may not take case
marking (there is no \N{*tsawl}).  However, the stem \N{tsa-} may be
used with the case endings, \N{tsal, tsar}, etc., again in rapid
speech.
\label{morph:pron:tsa}\index{tsaw@\textbf{tsaw}}\index{tsa'u@\textbf{tsa'u}}
\LNWiki{6/5/2010}{https://wiki.learnnavi.org/Canon/2010/March-June\%23History_of_Tsaw}
\NTeri{3/8/2011}{http://naviteri.org/2011/08/new-vocabulary-clothing/comment-page-1/\#comment-917}

\subsubsection{} The reflexive pronoun \N{sno} is not altered for
number. \index{sno@\textbf{sno}!not marked for number}

\subsubsection{} The third person animate indefinite pronoun is
\N{fko} (gen.\ \N{fkeyä}).
\LNWiki{17/5/2013}{https://wiki.learnnavi.org/Canon/2013\%23Double_Dative_and_more}

\subsection{Ceremonial/Honorific Pronouns}

\begin{center}
\begin{tabular}{rllll}
      & Singular & Dual & Trial & Plural \\ 
\hline
1 exclusive   & \N{\ACC{o}he}  & \N{\ACC{mo}he}  & \N{\ACC{pxo}he}   & \N{ay\ACC{o}he} \\
2nd       & \N{nge\ACC{nga}} & \N{menge\ACC{nga}} & \N{pxenge\ACC{nga}} & \N{aynge\ACC{nga}} \\
\end{tabular}
\end{center}\index{pronouns!honorific}\label{morph:hon-pron}

\subsubsection{} For the inclusive first person forms, use separate
pronouns, \N{ohe ngengasì} (with enclitic \N{sì} \E{and}).
\QUAESTIO{In the film we apparently get \N{ohengeyä}.}

%\QUAESTIO{\subsection{-po} We have \N{'awpo} and \N{fìpo}.  What about \N{tsapo}?}

\subsection{Lahe} When used as a pronoun, the adjective \N{lahe}
\E{other} has an irregular dative plural \N{aylaru}.
\index{lahe@\textbf{lahe}!declension}\label{morph:lahe:dat-pl}



\section{Prenouns}

\noindent The prenouns are adjective-like noun prefixes. \index{prenoun}

\subsection{Fì-} This prenoun is for proximal deixis, \E{this.}  When
it is followed by the plural prefix \N{ay+} they generally contract
into \N{fay+}, \E{these} in casual speech.  However, in precise or
formal speech, \N{fìay+} may be used, \Npawl{oel foru fìaylì'ut tolìng
a krr, kxawm oe harmahängaw}. \label{morph:prenoun:fi}
\index{fiì-@\textbf{fì-}} \index{fay+@\textbf{fay+}}
\LNForum{27/7/2013}{https://forum.learnnavi.org/language-updates/navi-details-from-avatarmeet-2013/}

\subsubsection{} Some nouns and adjectives pair with \N{fì-} to form
adverbs, such as \N{fìtrr} \E{today} and \N{fìtxan} \E{so (much)}.

\subsection{Tsa-} This is distal deixis, \E{that.}  When it is
followed by the plural prefix \N{ay+} they contract into \N{tsay+}
\E{those}. 
\index{tsa-@\textbf{tsa-}} \index{tsay+@\textbf{tsay+}}

\subsection{-Pe+} \label{morph:pre:pe} This question prenoun means
\E{what, which} as in \N{pelì'u} \E{which word?}  It is unusual in
that it may be either a prefix (\N{pelì'u}) or a suffix (\N{lì'upe}).
When prefixed, the following word takes lenition.  When the prefix is
followed by the plural prefix \N{ay+} they contract into \N{pay+}.

\subsection{Fra-} This prenoun means \E{all, every.}  When
it is followed by the plural prefix \N{ay+} they contract into \N{fray+}.
\index{fra-@\textbf{fra-}} \index{fray+@\textbf{fray+}}
\LNForum{27/7/2013}{https://forum.learnnavi.org/language-updates/navi-details-from-avatarmeet-2013/}

\subsection{Fne-} This prefix means \E{type (of), sort (of)}.
\index{fne-@\textbf{fne-}}

\subsubsection{} The prefix is related to the noun \N{fnel}, also
meaning \E{type, sort.}  It can occur with a noun in the genitive to
get the same meaning as the prefix.  \N{Tsafnel syulangä} and
\N{tsafnesyulang} both mean \E{that kind of flower}.
\index{fnel@\textbf{fnel}}

\subsection{Contraction} When a prenoun ends with the same vowel the
following word starts with, the vowels contract, as in \N{tsatan}
\E{that light} from \N{tsa-atan} (\horenref{l-and-s:phonotactics:precontract}).
\index{prenoun!contraction}\label{morph:prenoun:contraction}

\subsection{Combinations} The prenouns may combine on a single word,
in this order --- \index{prenoun!combinations}

\begin{center}
\begin{tabular}{cccccc}
\N{fì-} \\
\N{tsa-} & \N{fra-} & number marking & \N{fne-} & the noun & \N{-pe} \\
\N{pe+}
\end{tabular}
\end{center}

\noindent Only one from each column may be used, and of course the
question affix is only used once.  \QUAESTIO{The full details of this
ordering are not yet confirmed for \N{fra-}.}

\subsubsection{} Short plurals (\horenref{morph:short-plural}) are not
used with the deictic prenouns; \N{tsaytele} \E{those matters}, never
*\N{tsatele} (singular \N{txele}). \index{plural!short!not used with prenouns}


\section{Correlatives}

\noindent Demonstrative pronouns and certain common adverbs of time,
manner and place, are simply nouns paired with prenouns.  However, the
system is not perfectly regular.

% This is what I get for attaching several words to the same footnote.
\addtocounter{footnote}{1}
\newcounter{coraccent}\setcounter{coraccent}{\value{footnote}}

\begin{center}
\begin{tabular}{rllllll}% name person thing time place manner
 & Person & Thing & Action & Time & Place & Manner \\
\hline
\multirow{2}{*}{this} & \N{\ACC{fì}po} & \N{fì\ACC{'u}} &
  \N{fì\ACC{kem}} & \N{set} & \N{fì\ACC{tseng}(e)} & \N{fì\ACC{fya}}  \\ 
 & \E{this one} & \E{this (thing)} & \E{this (action)} & \E{now} &
  \E{here} & \E{thus} \\
\multirow{2}{*}{that} & \N{\ACC{tsa}tu} & \N{\ACC{tsa}'u} & \N{tsakem}\footnotemark[\value{coraccent}] & \N{tsa\ACC{krr}} &
   \N{tsatseng}\footnotemark[\value{coraccent}] & \N{\ACC{tsa}fya} \\
 & \E{that one} & \E{that (thing)} & \E{that (action)} & \E{then} &
  \E{there} & \E{in that way} \\
\multirow{2}{*}{all} & \N{\ACC{fra}po} & \N{\ACC{fra}'u} & --- &
  \N{\ACC{fra}krr} & \N{\ACC{fra}tseng} & \N{\ACC{fra}fya}  \\
 & \E{everyone} & \E{everything} &  & \E{always} & \E{everywhere} &
  \E{in every way} \\
\multirow{2}{*}{no} & \N{\ACC{kaw}tu} & \N{\ACC{ke}'u} & \N{\ACC{ke}kem} &
  \N{\ACC{kaw}krr} & \N{\ACC{kaw}tseng} & --- \\
 & \E{no one} & \E{nothing} & \E{no action} & \E{never} & \E{nowhere} \\
\end{tabular}
\end{center}\label{morph:correlatives}
\footnotetext[\value{coraccent}]{May be accented on either syllable.}
\LNWiki{18/5/2011}{http://wiki.learnnavi.org/index.php/Canon/2011/April-December\%23Kawtseng.2C_tsapo_and_prefixes}
\NTeri{24/7/2011}{http://naviteri.org/2011/07/txantsana-ultxa-mi-siatll-great-meeting-in-seattle/comment-page-1/\%23comment-845} % kekem

\subsubsection{} \QUAESTIO{Plurals for these are a bit funky.  Though
\N{tsa'u} is from \N{tsa-} and \N{'u}, the plural is \N{(ay)sa'u}.
Confirmed, but details might be nice.  How to work in \N{tsapo}?}

\subsubsection{} For the forms of \N{tsa'u}, see \horenref{morph:pron:tsa}.

\subsection{Questions} As with nouns, the question affix \N{-pe+} may
be either a leniting prefix or a suffix.

\begin{center}
\begin{tabular}{rl}
who? & \N{pe\ACC{su}}, \N{\ACC{tu}pe} \\
what (thing)? & \N{pe\ACC{u}}, \N{\ACC{'u}pe} \\
what (action)? & \N{pe\ACC{hem}} \N{\ACC{kem}pe} \\
when? & \N{pe\ACC{hrr}}, \N{\ACC{krr}pe} \\
\end{tabular}
\hskip2em
\begin{tabular}{rll}
where? & \N{pe\ACC{seng}}, \N{\ACC{tseng}pe} \\
how? & \N{pe\ACC{fya}}, \N{\ACC{fya}pe} \\
why? & \N{pe\ACC{lun}}, \N{\ACC{lum}pe} \\
what kind (of)? & \N{pe\ACC{fnel}}, \N{\ACC{fne}pe}\\
\end{tabular}
\end{center}

\subsubsection{} The question word for people, \N{tupe} / \N{pesu}
\E{who}, has a enormous collection of gendered and non-singular forms:

\begin{center}
\begin{tabular}{lccc}
 & Common & Male & Female \\
\hline
Singular & \N{pe\ACC{su}}, \N{\ACC{tu}pe} & 
           \N{pe\ACC{stan}}, \N{tu\ACC{tam}pe} &
           \N{pe\ACC{ste}}, \N{tu\ACC{te}pe} \\
Dual     & \N{pem\ACC{su}}, \N{me\ACC{su}pe} & 
           \N{pem\ACC{stan}}, \N{me\ACC{stam}pe} &
           \N{pem\ACC{ste}}, \N{me\ACC{ste}pe} \\
Trial    & \N{pep\ACC{su}}, \N{pxe\ACC{su}pe} & 
           \N{pep\ACC{stan}}, \N{pxe\ACC{stam}pe} &
           \N{pep\ACC{ste}}, \N{pxe\ACC{ste}pe} \\
Plural   & \N{pay\ACC{su}}, \N{(ay)\ACC{su}pe} & 
           \N{pay\ACC{stan}}, \N{(ay)\ACC{stam}pe} &
           \N{pay\ACC{ste}}, \N{(ay)\ACC{ste}pe} \\
\end{tabular}
\end{center}

\noindent The non-singular forms of \N{pehem} / \N{kempe} follow a
similar pattern:

\begin{center}
\begin{tabular}{lc}
Singular & \N{pe\ACC{hem}}, \N{\ACC{kem}pe} \\
Dual & \N{pem\ACC{hem}}, \N{me\ACC{hem}pe} \\
Trial & \N{pep\ACC{hem}}, \N{pxe\ACC{hem}pe} \\
Plural & \N{pay\ACC{hem}}, \N{(ay)\ACC{hem}pe} \\
\end{tabular}
\end{center}

\noindent\NTeri{30/7/2011}{http://naviteri.org/2011/07/number-in-na’vi/}

\subsection{Fì'u and Tsaw in Clause Nominalization} The demonstrative
pronoun \N{fì'u} and inanimate pronoun \N{tsaw} are used with the
attributive particle \N{a} to nominalize clauses
(\horenref{syn:clause-nom}).  When the attributive particle follows
certain case forms of the pronoun, they contract:\label{morph:fwa-tsawa}
% http://forum.learnnavi.org/language-updates/txelanit-hivawl/

\begin{center}
\begin{tabular}{rcc}
Case & \N{Fì'u} Contraction & \N{Tsaw} Contraction \\
\hline
Subjective & \N{fwa} ($<$ \N{fì'u a}) & \N{\ACC{tsa}wa} \\
Agentive & \N{\ACC{fu}la} ($<$ \N{fì'ul a}) & \N{\QUAESTIO{tsala}} \\
Patientive & \N{\ACC{fu}ta} ($<$ \N{fì'ut a}) & \N{\ACC{tsa}ta} \\
Topical & \N{\ACC{fu}ria} ($<$ \N{fì'uri a}) & \N{\ACC{tsa}ria} \\
\end{tabular}
\end{center}
\index{fwa@\textbf{fwa}}\index{tsawa@\textbf{tsawa}}
\index{fula@\textbf{fula}}
\index{futa@\textbf{futa}}\index{tsata@\textbf{tsata}}
\index{furia@\textbf{furia}}\index{tsaria@\textbf{tsaria}}
\LNWiki{18/6/2010}{http://wiki.learnnavi.org/index.php/Canon/2010/March-June\%23The_contrast_between_fwa.2Ftsawa.2C_furia.2Ftsaria}

\subsection{Fmawn and Tì'eyng in Clause Nominalization} While \N{fì'u}
and \N{tsaw} may nominaize clauses of most types, verbs of hearing,
speaking and questioning prefer the nouns \N{fmawn} \E{news}, \N{tì'eyng}
\E{answer} and \N{faylì'u} \E{these words}.  There are fewer
contractions: \label{morph:fmawn} 

\begin{center}
\begin{tabular}{rc}
Case & Contraction \\
\hline
Subjective & \N{teynga} ($<$ \N{tì'eyng a}) \\
Agentive & \N{teyngla} ($<$ \N{tì'eyngìl a}) \\
Patientive & \N{teyngta} ($<$ \N{tì'eyngit a})
\end{tabular}
\end{center}
\index{fmawnta@\textbf{fmawnta}} \index{teynga@\textbf{teynga}}
\index{teyngla@\textbf{teyngla}} \index{teyngta@\textbf{teyngta}}
\index{fmawn@\textbf{fmawn}} \index{tiì'eyng@\textbf{tì'eyng}}
\index{fayluta@\textbf{fayluta}} \index{fayliì'u@\textbf{faylì'u}}

\noindent There are contractions only in the patientive for \N{fmawn}
and \N{faylì'u}, which are \N{fmawnta} ($<$ \N{fmawnit a}) and
\N{fayluta} ($<$ \N{faylì'ut a}).  See \horenref{syn:quot:nominalized}
for the syntax.
\NTeri{31/8/2011}{http://naviteri.org/2011/08/reported-speech-reported-questions/}


\section{The Adjective}
\subsection{Attribution} Attributive adjectives are joined to their
noun with the affix \N{-a-}, which is attached to the adjective on the
side closest to the noun, as in \N{yerik awin} or \N{wina yerik} for
``a fast yerik.''\label{morph:adj-attr}
\index{-a-@\textbf{-a-}\index{adjective!attributive!formation}}

\subsubsection{} A derived adjective in \N{le-} usually drops the
prefixed (but not suffixed) \N{a-}, so either \N{ayftxozä lefpom} or,
more rarely, \N{ayftxozä alefpom}.  However, when the \N{le-}adjective
comes before the noun, it will always have the attributive \N{-a-},
\N{lefpoma ayftxozä}.

\begin{center}
\begin{tabular}{ll}
\N{ayftxozä lefpom} & usual \\
\N{ayftxozä \uwave{a}lefpom} &  permitted \\
$*$\N{lefpom ayftxozä} &  an error \\
\N{lefpom\uwave{a} ayftxozä} &  correct \\
\end{tabular}
\end{center}


\section{The Verb}
\subsection{Infix Location} Frommer describes three positions for verb
infixes: pre-first position, first position and second position.  Each
position has infixes of a particular type (described below).

\subsubsection{} All infixes occur in the last (ultima) and
next-to-last (penult) syllables of the verb stem, and are inserted
before the vowel, diphthong or pseudovowel of that syllable, as in
\N{kä} $>$ \N{k‹ìm›ä} and \N{taron} $>$ \N{t‹ol›ar‹ei›on}.

\subsubsection{} If a syllable has no onset consonant(s) the infix
still precedes the vowel, as in \N{omum} $>$ \N{‹iv›omum} and \N{ftia}
$>$ \N{fti‹ats›a}.

\subsubsection{} The stress accent stays on the vowel that originally
had it before any infixes were added, \N{\ACC{haw}nu} $>$
\N{h‹il\ACC{v›aw}nu}.\footnote{Exception: the verb \N{o\ACC{mum}}
shifts the accent to the \N{o} for any inflected or derived forms,
\N{i\ACC{vo}mum}, \N{nìaw\ACC{no}num}. \index{omum@\textbf{omum}!accenting}
The verb \N{i\ACC{nan}} follows the same pattern, \N{o\ACC{li}nan}.
\index{inan@\textbf{inan}!accenting}
}

\subsubsection{} Usually, infixes are placed only in one element of a
compound verb.  For example, \N{yom-tìng} \E{feed} is a compound of
\N{yom} \E{eat} and \N{tìng} \E{give}.  The perfective of this is not
*\N{y\INF{ol}omtìng}, but \N{yomt\INF{ol}ìng}.  Most compound verbs
will have the verb element last, which will take the infixes.  A few
compounds, however, do add infixes to the first element.  These must
be learned from the lexicon.  \index{verb!compound!infix location}

\subsubsection{} A small number of verb+verb compounds take infixes in
both elements of the compound, such as \N{kan'ìn} \E{specialize in},
made up of \N{kan} \E{aim, intend} and \N{'ìn} \E{be busy}.
\Ultxa{2/10/2010}{http://wiki.learnnavi.org/index.php/Canon/2010/UltxaAyharyu\%C3\%A4\%23Transitivity_and_Infix_Positions}

\subsection{Pre-first Position} These infixes change transitivity.
They are inserted before the vowel of the next-to-last syllable of a
verb, or the verb syllable if the verb has only one syllable.
\label{morph:pre-first}
\index{reflexive!formation}\index{causative!formation}

\begin{center}
\begin{tabular}{lr}
Causative & \N{\INF{eyk}} \\
Reflexive & \N{\INF{äp}} \\
\end{tabular}
\end{center}

\noindent\LNWiki{1/2/2010}{http://wiki.learnnavi.org/index.php/Canon\%23Reflexives_and_Naming} % reflexive
\LNWiki{15/2/2010}{http://wiki.learnnavi.org/index.php/Canon\%23Trials_.26_transitivity} % causative

\subsubsection{} In casual conversation the reflexive perfective of
\N{si}-construction verbs, \N{säpo\ACC{li}}, is often pronounced
\N{spo\ACC{li}}.
\NTeri{3/8/2011}{http://naviteri.org/2011/08/new-vocabulary-clothing/}

\subsubsection{} The causative reflexive, ``cause oneself to,'' is
formed with \N{\INF{äp}\INF{eyk}}, so \N{po täpeykerkup} \E{he causes
himself to die}.  \index{reflexive!of a causative}

\subsection{First Position} These mark tense, aspect and mood, and
create participles.  They are inserted before the vowel of the
next-to-last syllable of a verb, or the verb syllable if the verb has
only one syllable.  They will always follow any pre-first position
infixes. \label{morph:verb:first-position}

\begin{center}
\begin{tabular}{r|ccc}
 & Tense only & Perfective & Imperfective \\
\hline
Future & \N{\INF{ay}, \INF{asy}} & \N{\INF{aly}} & \N{\INF{ary}} \\
Near future & \N{\INF{ìy}, \INF{ìsy}} & \N{\INF{ìly}} & \N{\INF{ìry}} \\
General    &  — & \N{\INF{ol}} & \N{\INF{er}} \\
Near past & \N{\INF{ìm}} & \N{\INF{ìlm}} & \N{\INF{ìrm}} \\
Past & \N{\INF{am}} & \N{\INF{alm}} & \N{\INF{arm}} \\
\end{tabular}
\end{center}
\LanguageLog{}
\LNWiki{27/1/2010}{http://wiki.learnnavi.org/index.php/Canon\%23Extracts_from_various_emails}
\LNWiki{19/2/2010}{http://wiki.learnnavi.org/index.php/Canon\%23Evidential} %ìrm

\subsubsection{} The futures with \N{s} mark intention
(\horenref{syn:verb:intenfut}).

\subsubsection{} The subjunctive infix, \N{\INF{iv}}, has a restricted
set of combinations with fewer tense gradations.
\index{subjunctive!infix}

\begin{center}
\begin{tabular}{r|ccc}
         & Tense only & Perfective & Imperfective \\
\hline
Future & \N{\INF{ìyev}, \INF{iyev}} & — & — \\
General & \N{\INF{iv}} & \N{\INF{ilv}} & \N{\INF{irv}} \\
Past & \N{\INF{imv}} & — & —
\end{tabular}
\end{center}

\noindent\LNWiki{9/1/2010}{http://wiki.learnnavi.org/index.php/Canon\%23Extracts_from_various_emails}
\LNWiki{30/1/2010}{http://wiki.learnnavi.org/index.php/Canon\%23Vocabulary_and_.3Ciyev.3E}
\LNWiki{30/1/2010}{http://wiki.learnnavi.org/index.php/Canon\%23Fused_-iv-_Infixes}

\subsubsection{} There are only two participle infixes.  They do not
combine with tense, aspect or mood infixes. \index{participle!formation}

\begin{center}
\begin{tabular}{lr}
Active & \N{\INF{us}} \\
Passive & \N{\INF{awn}} \\
\end{tabular}
\end{center}

\noindent Since the participles are adjectives that cannot be used as
predicates, they will always occur with the attributive adjective
affix \N{-a-} (\horenref{morph:adj-attr}, \horenref{syn:part:attr}).
\LNWiki{13/3/2011}{http://wiki.learnnavi.org/index.php/Canon/2010/March-June\%23Participial_Infixes}


\subsection{Second Position} These infixes, which indicate speaker
affect or judgement, occur in the final syllable of the verb, or after
the first position infixes in a verb of one syllable.
\index{attitude infixes}\label{morph:verb:2nd-pos}

\begin{center}
\begin{tabular}{rl}
Positive attitude & \N{\INF{ei}}, \N{\INF{eiy}} (\horenref{l-and-s:eiy-epenth}) \\
Negative attitude & \N{\INF{äng}}, \N{\INF{eng}} (\horenref{l-and-s:eng}) \\
Formal, ceremonial & \N{\INF{uy}} \\
Inferential, suppositional & \N{\INF{ats}} \\
\end{tabular}
\end{center}

\noindent\LNWiki{19/2/2010}{http://wiki.learnnavi.org/index.php/Canon\%23Evidential} % <ats>

\subsection{Examples} The rules given above are a bit abstract, so I
give here examples of some possible inflections for a few verb shapes.
The verbs are \N{eyk} \E{lead} as an example of a single-syllable word
with no onset consonant, \N{fpak} \E{stop} as a single-syllable with
consonant cluster onset, \N{taron} \E{hunt} the usual two-syllable
word Frommer uses in examples, and \N{yom·tìng} \E{feed}, a compound
verb, in which only the final element is inflected.

\begin{center}\small
\begin{tabular}{lllll}
           & \N{eyk} & \N{fpak} & \N{\ACC{ta}ron} & \N{\ACC{yom}·tìng} \\
\hline
Near past & \N{ì\ACC{meyk}} & \N{fpì\ACC{mak}} & \N{tì\ACC{ma}ron} & \N{\ACC{yom}tìmìng} \\
Reflexive  & \N{ä\ACC{peyk}} & \N{fpä\ACC{pak}} & \N{tä\ACC{pa}ron} & \N{\ACC{yom}täpìng} \\
Refl., near past & \N{äpì\ACC{meyk}} & \N{fpäpì\ACC{mak}} & \N{täpì\ACC{ma}ron} & \N{\ACC{yom}täpìmìng} \\
Ceremonial & \N{u\ACC{yeyk}} & \N{fpu\ACC{yak}} & \N{\ACC{ta}ruyon} & \N{\ACC{yom}tuyìng} \\
Perf., cerem. & \N{olu\ACC{yeyk}} & \N{fpolu\ACC{yak}} & \N{to\ACC{la}ruyon} & \N{\ACC{yom}toluyìng} \\
Refl., perf., cerem. & \N{äpolu\ACC{yeyk}} & \N{fpäpolu\ACC{yak}} & \N{täpo\ACC{la}ruyon} & \N{\ACC{yom}täpoluyìng} \\
\end{tabular}
\end{center}

\noindent The meanings of some of these examples stretch good sense to
the breaking point.  The purpose of these is only to show infix
locations across a consistent set of verb shapes.
